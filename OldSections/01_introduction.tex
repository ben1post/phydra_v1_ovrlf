\documentclass[template.tex]{subfiles}

\begin{document}
\introduction  %% \introduction[modified heading if necessary]
% introduction should be adundantly clear with no unecessary details!
% can also mention open source, but just high-level summaries, same on phydra details, no detail in intro
% establish, why what you make is different from everything done before. why it is important! (2 sentences)

% 2nd Section:
% Background, theoretical framework. Specifics here!

\textbf{ONLY ROUGH POINTS; NOT FORMULATED YET}


\subsection{Biogeochemical role of phytoplankton}
%- P1: why phytoplankton important?
- major primary producers ocean ecosystem, understanding important, global change (important part of climate models)

- complex system --> complex models?

- phytoplankton

\subsection{Phytoplankton modeling: From NPZD to DARWIN}
%- P2: why people model phytoplankton? (what are the types of model we use? npzd, pft, size, darwin, etc.)
quote Gentleman history of marine ecosystem modeling
\\
- cite the beautiful paper on Steele's legacy
\citep{Anderson2019RememberingEcosystems}


\\


\subsection{Running before we can walk}
%- P3: all these different models, what is the problem? [Motivation for a better toolkit]
The simplicity of an NPZD model, where each of the four ecosystem components is represented by a single state variable, can be criticised as oversimplifying the complexity of a natural plankton community.

Actually can reference all these papers critical of modeling:
- Running before we can walk, misleading legacies, Perplexing parameterisations

and state: phydra is built with the idea in mind, that all ecosystem models are theories/hypotheses, that need to be tested against data, and tested against other models (particularly models of varying complexity). Occam's razor. 

- lots of models, in lots of (often old) programming languages, varying implementations
- all different, hidden (+very complex) codebase, not easy to adapt \& modify

ALTHOUGH, actually often very similar in nature and underlying technical processes, so if we could use a common framework that is sufficiently accessible and flexible, we can spend much less time on technicalities.

- but development in data science & other computer science areas, towards reproducible tools, towards shared frameworks and capabilities (like Benoît said in his blog post)
- tools should foster scientific collaboration
"easy to use, open source, easy to modify, toolbox"
- python Pangeo software ecosystem (jupyter, xarray, zarr)

\subsection{collaborative modelling using open source solution}
%- P4: Here's my solution! Here I present flexible model framework. explain basic structure and function

quick intro, and set aim!

There have been different projects aimed at increasing the ease of model creation, both on the lower and higher-end

All projects have trade-offs, and what is most important is a common language and communication within the user community.
There are these.. and these projects (NAME SOME EXAMPLES).


Why python? -> Pangeo, rising language, relatively easy to learn


Why Phydra? -> collaborative development and open science are central to the Python project, no such previous tool available, but seems very much needed (see previous problems!)

% AIM for phydra:
Many large physical and ecological models are to this day written in FORTRAN. Due to the statically-typed nature it is more computationally efficient particularly for larger models. Yet despite it's speed, FORTRAN remains less accessible to scientists and usage and literacy of Python in the scientific community far exceeds that of FORTRAN.
Phydra is a project to bridge the gap between older, but more efficient ecosystem model code written in low-level languages like C++ and FORTRAN and the flexibility and usability of modern high-level programming languages. Python is a free and open-source language that is flexible, popular and has gained wide use in the scientific community. By being written in Python, Phydra can interact with a rich environment of popular scientific and numerical packages. 


"xarray-simlab /("Phydra") is a tool for fast model development and easy, interactive model exploration. It aims at empowering scientists to do better research in less time, collaborate efficiently and make new discoveries."

Pangeo project (e.g. Xarray, Dask, Zarr, Jupyter). 

% AIM for this paper! qualify limits

Here, we demonstrate the usage of the phydra v1 package in a simple NPZD setting and to showcase the flexibility, in a more complex size based trophic structure model. Both in slab physics.
The package can be used for any model setting, so far only slab physics are hard-coded in package.
Ultility of phydra v1 is showcased in a parameter fitting for Use Case I
Flexibility is showcased in Use Case II, where physical setting is kept the same, but multiple FTs added.
Finally, the relevance of this work and potential next steps are discussed.

//
The utility of the phydra package is here shown via three model implementations. Two models from literature are recreated, and the flexible nature of this package allows a combination of as a third example. 

Here, we demonstrate the usage of the phydra v1 package in a simple NPZD setting and to showcase the flexibility, in a more complex size based trophic structure model. 

"In this paper, we (a) describe modeling approach, (b) apply modeling approach in 3 examples"

can also mention open source, but just high-level summaries, same on phydra details, no detail in intro

establish, why what you make is different from everything done before. why it is important! (2 sentences)


\end{document}




