\documentclass[template.tex]{subfiles}


\begin{document}


\subsection{Use Case 2}

For the second use case we have implemented a size-spectral NPZ ecosystem model as presented in Figure \ref{Figure:phydraschematics_1} (b) with a nutrient $N$ (in this case nitrate), multiple size classes of both phytoplankton $P_i$ and zooplankton $Z_j$ as state variables. 
This model use case employs a simple chemostat as physical setting. A flow-through culture, where organisms grow in a bottle with a continuous influx of a nutrient solution. At the same rate all components of the bottle flow out of the system, so that volume remains constant over time.
Model structure and parameterisation is adapted from the ASTroCAT model \citep{Banas2011b}, with slight modifications to the physical environment and parameterisation. 
\\
The model describes a size-structured community of phytoplankton and zooplankton, with each state variable defined by their equivalent spherical diameter (ESD). In model runs presented in this paper, we follow \citet{Banas2011b} in running simulations with 40 size classes of equally log-spaced ESD of $P$ (1 to 20 \unit{\mu m}), and 40 matching classes of $Z$ (2.1 to 460  \unit{\mu m}). 
The model can be defined with any number of size classes within meaningful boundaries of the allometric parameterisation. Size classes are denoted by the subscript $i$ for phytoplankton and $j$ for zooplankton. In the implementation of a size-spectral model by \citeauthor{Banas2011b} the food-web is implicitly made up of pairs of $P_i$ and $Z_j$, to avoid creating classes at the ends of the size spectrum that are artificially released or suppressed by grazing pressure.

The main modification to the model structure in ASTroCAT is the physical setting. An additional outflow acts as a loss term on all model components to place them in a traditional experimental setting for phytoplankton ecology, a chemostat system.

\subsubsection{Flow}

Nutrient supply to the system is defined by a linear flow rate $f$ and the nutrient concentration in the supplied solution $N^\emptyset$. All components, including phytoplankton and zooplankton are lost from the system at the same rate $f$.

\subsubsection{Nutrient}
Dissolved inorganic nitrogen (DIN) in the model system ($N$) is supplied from influx of medium (with nutrient concentration $N^\emptyset$) at a constant rate $f$ and the fraction of grazed biomass that is excreted by $Z$. The concentration $N^\emptyset$ and the flow rate $f$ determine the total nutrient supply to the system. At the same rate $N$ is flowing out from the system. Zooplankton excretion is a recycled term, that also adds back into the pool of $N$. The major loss term to $N$ is phytoplankton growth.

\subsubsection{Phytoplankton}
Phytoplankton biomass $P$ increases through  nutrient-limited growth. Nutrient limitation of phytoplankton growth $\gamma_i^N$ is described by the Michaelis-Menten (or Monod) equation:

\begin{equation}
    \gamma_i^N =  \frac{N}{k_i^N + N} 
\end{equation}

where $k_i^N$ is the size-dependent half-saturation constant. $N$ is ambient nutrient concentration in the medium.

Non-grazing mortality of phytoplankton is described the factor $m^P$ that is scaled by the maximum intrinsic growth rate $\mu_i^{\emptyset}$. This accounts for natural mortality and excretion.

\subsubsection{Zooplankton}
Zooplankton size class $j$ grazing on phytoplankton size class $i$ is calculated by
\begin{equation}
    G_{ij}^P = \mu_j^Z \ \frac{ \phi_{ij} \cdot P_i }{ k_j^Z + \sum_{i}(\phi_{ij} \cdot P_i) } \ Z_j
\end{equation}
where $\mu_j^Z$ is the size-dependent maximum ingestion rate, $k_j^Z$ is the prey half-saturation level and $\phi_{ij}$ is the relative preference of $Z_j$ for prey type $P_i$.\\

Prey preference is assumed to vary with phytoplankton size $size_i^{P}$ in a log-Gaussian distribution around an optimal prey size for each grazer $size_j^{opt}$.

\begin{equation}
    \phi_{ij} = exp \left[ -\left( \ \frac{ log_{10}(size_i^{P}) - log_{10}(size_j^{opt}) }{ \Delta size^{P} } \right) \right]
\end{equation}
Where $\Delta size^{P}$ is the prey size tolerance parameter, with units of \unit{log_{10}(\mu m)}, that controls the width of the Gaussian distribution.\\

Zooplankton growth is a product of total biomass grazed ($G^P$) and the gross growth efficiency (GGE) of zooplankton. The two parameters defining GGE in this model are absorption efficiency ($\beta$) and net production efficiency ($\epsilon$). Adsorption efficiency $\beta$ describes the fraction of $G^P$ which is absorbed in the gut, of which the fraction $\epsilon$ is actually assimilated into biomass (to $Z$: \ $\beta \epsilon$), while the rest is excreted as DIN (to $N$: \ $\beta (1-\epsilon)$). GGE specifically is the product of $\beta$ and $\epsilon$, for which values between 0.2 and 0.3 have been observed for a wide range of zooplankton  \citep{Straile1997GrossGroup}. The fraction of $G^P$ egested (e.g. as faecal pellets) is calculated via $1-\beta$, which in the current model setup is assumed to be exported quickly in the flow-through setting and lost from the system. 

Zooplankton experiences quadratic mortality according to the parameter $m^{Z2}$, describing higher-order mortality and predation of zooplankton that is removed from the system \citep{Edwards2000TheModels}. This closure term is quadratic, calculated with the implicit assumption that mortality of $Z_j$ is proportional to total zooplankton biomass $\sum_{j} Z_j$.

\clearpage
\subsubsection{Model equations}
The rates of change of the state variables are described by the following set of equations. For the definition of all symbols used here see Table \ref{appendix:table:usecase2symbols}.
See \citet{Banas2011b} for a more detailed discussion of model structure and formulation. 

\begin{equation}
    \frac{d N}{d t} = 
    f \ N^\emptyset % Nutrient mixing
    +  \beta (1 - \epsilon) \sum_{j} \sum_{i} G_{ij}^P % Unassimilated grazing by Z
    - \sum_{i} ( \mu_i^{\emptyset} \ \gamma_i^N \ P_i) % Phytoplankton gains
    - f \ N
\end{equation}

%PHYTOPLANKTON
\begin{equation}
    \frac{d P_i}{d t} =
    \mu_i^{\emptyset} \  \gamma_i^N \   P_i  % Phytoplankton gains
    - m^P  \ \mu_i^{\emptyset} \ P_i % Linear mortality
    - \sum_{j} G_{ij}^P % Z grazing
    - f \ P_i
\end{equation}

%ZOOPLANKTON
\begin{equation}
    \frac{d Z_j}{d t} =
    \beta \ \epsilon \ \sum_{i} G_{ij}^P % Assimilated grazing
    - m^{Z2} \ Z_j \ \sum_{j} Z_j  % Quadratic mortality
    - f \ Z_j
\end{equation}




\clearpage
%TABLES & FIGURES


\begin{table*}[t]

\caption{ Definition of symbols employed in use case 2 appendix. (\unit{\mu M \ N} = \unit{mmol \ Nitrogen \ m^{-3}}) }

\begin{tabular}{l l l}
Symbol & Meaning & Unit\\
\tophline
\tophline
State variables:\\
\middlehline
$N$ & concentration of nutrient in model system & \unit{\mu M \ N} \\
$P_i$ & concentration of phytoplankton size class $i$ biomass  & \unit{\mu M \ N} \\
$Z_j$ & concentration of zooplankton size class $j$ biomass  & \unit{\mu M \ N} \\


Forcings:\\
\middlehline
$f$ & flow-through rate of chemostat environment (influx \& outflux) & \unit{d^{-1}} \\
$N_0$ & nutrient concentration of source medium for chemostat & \unit{\mu M \ N} \\



Processes:\\
\middlehline
$\gamma_i^N$ & nutrient-dependence of the growth of phytoplankton size class $i$ & dimensionless\\
$G_{ij}^P$ & total biomass of phytoplankton size class $j$ grazed by zooplankton size class $i$ & \unit{\mu M \ N} \\
$\phi_{ij}$ & Preference of grazer $Z_j$ for prey $P_i$ & dimensionless \\


Parameters: \\
\middlehline
$\mu_i^{\emptyset}$ & size-dependent max. growth rate of phytoplankton & \unit{d^{-1}} \\
$k_i^N$ & Phytoplankton size class $i$ half-saturation constant for nutrient uptake & \unit{\mu M \ N} \\
$m^P$ & fraction of size-dependent $\mu_i^{\emptyset}$ that is phytoplankton mortality & dimensionless \\

$\mu_j^Z$ & maximum ingestion rate for zooplankton size class $j$ &  \unit{d^{-1}} \\

$k_j^Z$ & Prey half-saturation level for zooplankton size class $j$ & \unit{\mu M \ N} \\
$size_i^{P}$ & Individual size of phytoplankton (prey) size class $i$ & \unit{\mu m} \\
$size_j^{opt}$ & Optimal prey size for zooplankton size class $j$ & \unit{\mu m} \\
$\Delta size^{P}$ & Prey size tolerance for grazers & \unit{log_{10}(\mu m)} \\

$\beta$ & absorption efficiency of zooplankton, fraction of $G_{ij}^P$ &  dimensionless \\
$\epsilon$ & net production efficiency of zooplankton, fraction of $\beta$ & dimensionless \\

$m^{Z2}$ & quadratic mortality of zooplankton, dependent on total $Z$ biomass & \unit{(\mu M \ N)^{-1} \ d^{-1}} \\

%\middlehline
%\bottomhline
\end{tabular}
\label{appendix:table:usecase2symbols}
%\belowtable{This is a test} % Table Footnotes
\end{table*}

%%%%%%%%%%%%%%%%



\clearpage

% this is a custom function to be able to see references when rendering subfiles:
\biblio

\end{document}
