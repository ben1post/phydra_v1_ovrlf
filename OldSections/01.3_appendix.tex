\documentclass[template.tex]{subfiles}


\begin{document}


\subsection{Use Case 3}
For the third case the model structure is a combination of the previous two use cases. The result is a size-structured NPZD slab model, as presented in Figure \ref{Figure:phydraschematics_3} with a nutrient $N$, multiple size classes of both phytoplankton $P_i$ and zooplankton $Z_j$ and detritus $D$ as state variables.
This ecosystem is embedded in a slab physical setting, as per \citet{Evans1985ACycles}. The model ocean is built up of two layers. A biologically inert deep ocean is situated below a well mixed upper layer of variable depth that contains the ecosystem. \\

The general model structure is adapted from the EMPOWER model presented by \citet{Anderson2015c}. Modifications to the formulation, were aimed at simplifying the description of physical forcings and phytoplankton growth (see Section A 1.1 for further detail). We replace the singular $P$ and $Z$ state variables, with the diverse $P_i$ and $Z_j$ and their size-structured trophic web as defined Appendix 1.2 and adapted from \citet{Banas2011b}. The model is driven by empirical forcing describing the depth of the mixed layer ($H$), average temperature of the mixed layer ($T$), photosynthetically active radiation at the surface ($I$) and nutrient concentration in the deep layer ($N^\emptyset$).\\

The model ecosystem describes a size-structured community of phytoplankton and zooplankton, with each state variable defined by their equivalent spherical diameter (ESD). In model runs presented in this paper, we follow \citet{Banas2011b} in running simulations with 40 (or 20?) size classes of equally log-spaced ESD of $P$ (1 to 20 \unit{\mu m}), and 40 (?)  matching classes of $Z$ (2.1 to 460  \unit{\mu m}). 
Size classes are denoted by the subscript $i$ for phytoplankton and $j$ for zooplankton. In the implementation of a size-spectral model by \citeauthor{Banas2011b} the food-web is implicitly made up of pairs of $P_i$ and $Z_j$, to avoid creating classes at the ends of the size spectrum that are artificially released or suppressed by grazing pressure.



\subsubsection{Mixing}
% Nutrient dynamics
The zero-dimensional physical slab setting describes two vertical layers of which the deeper layer supplies nutrients to the upper layer, whilst other components are mixed to the deep layer and lost from the system.
The magnitude of mixing is described by the coefficient $K$:

\begin{equation}
    K = \frac{h^{+} + \kappa}{H}
\end{equation}

Constant diffusive mixing is parameterized by $\kappa$. Variable mixing is a function of the change in MLD over time $h = \frac{d}{d t} H$. The derivative of MLD ($h$) is positive when the mixed layer deepens. The function $h^{+}$ defines the effects of entrainment and detrainment due to the changes in MLD as $h^{+} = \max(0, \ h)$. When the mixed layer shallows, $h^{+}$ does not modify $K$ (i.e. returns 0 instead of a negative value), based on the assumption that detrainment of mass and the increase in concentration due to the reduced volume of the mixed layer are balanced \citep{Evans1985ACycles}. \\


\subsubsection{Nutrients}
Dissolved inorganic nitrogen in the mixed layer ($N$) is supplied via mixing, zooplankton excretion and detritus remineralisation.
Nutrients are entrained from the bottom layer. Mixing of nutrients is a positive term adding to $N$ along the gradient between $N^\emptyset$ and $N$. The general direction of transport is from a nutrient-rich bottom layer to the upper layer supporting phytoplankton growth, which is the only loss term.


\subsubsection{Phytoplankton}
Phytoplankton size-class biomass ($P_i$) increases through temperature-dependent, light- \& nutrient-limited growth. The growth rate ($\mu_i^{P}$) is the product of a size-dependent maximum growth rate ($\mu_i^{\emptyset}$) and growth-dependency on nutrients ($\gamma_i^{N}$), as well as the general growth dependencies  on temperature ($\gamma^{T}$), light ($\gamma^{I}$) that are shared between size classes: 

\begin{equation}
    \mu_i^{P} = \mu_i^{\emptyset} \ \gamma^{T} \ \gamma^{I} \ \gamma_i^{N}
\end{equation}

$T$ is the average temperature of the mixed layer in \unit{\degree C}, as supplied from model forcing. Temperature dependence of the growth rate ($\gamma^{T}$) is calculated via the Eppley curve \citep{Eppley1972TemperatureSea}.

\begin{equation}
    \gamma^{T} = \exp{(0.063 \ T)} \label{mumax}
\end{equation}

The light-limiting term $\gamma^{I}$ represents growth-dependence on total light ($I$) available to phytoplankton the upper mixed layer. We use a simplified form of Steele's formulation to describe light-limitation of phytoplankton growth in the mixed layer, as adapted from \citet{Acevedo-Trejos2016} and originally described in \citet{Steele1962EnvironmentalSea}.

\begin{equation}
    \gamma^{I} = \frac{1}{H} \int_{0}^{H}\left[ \frac{I(z)}{I^{opt}} \cdot \exp{\left( 1 - \frac{I(z)}{I^{opt}} \right) }  \right]dz \label{steele}
\end{equation}

Where $I^{opt}$ is the light level at which photosynthesis saturates and $I(z)$ is the irradiance at depth $z$.
The irradiance forcing $I$ is a temporally and spatially averaged monthly climatology of photosynthetically active radiation (PAR) at the surface. 

Attenuation of $I$ at depth $z$ in the mixed layer is calculated according to the Lambert-Beer equation:

\begin{equation}
    I(z) = I \ \exp{(-k^{PAR} \ z)} \label{beer}
\end{equation}

The attenuation coefficient $k^{PAR}$ is the sum of the attenuation coefficient of seawater $k^w$ and that of phytoplankton biomass $k^c$, which is multiplied by the total phytoplankton biomass $\sum_i P_i$:

\begin{equation}
    k^{PAR} = k^w + k^c \cdot \sum_i P_i
\end{equation}

Combining the equations and integrating across the mixed layer, the numerical solution for the integrated light-limiting term affecting phytoplankton growth is calculated. Integrated values within the mixed layer larger than the optimal irradiance will limit growth, to model effects of photo-inhibition \citep{Steele1962EnvironmentalSea}.\\

The optimal irradiance parameter ($I^{opt}$ ) has not been resolved along the size spectrum and is shared by all size classes $P_i$. Similarly temperature dependence of growth is not resolved between phytoplankton state variables. This was done to simplify model structure for presentation as a general use case. \\

Mortality of phytoplankton is described the factor $m^P$ that is a fraction of the size-dependent maximum intrinsic growth rate $\mu_i^{\emptyset}$. This linear mortality term accounts for natural mortality and excretion and is added to the pool of detritus $D$.

\subsubsection{Zooplankton}

Zooplankton size class $j$ grazing on phytoplankton size class $i$ is calculated by
\begin{equation}
    G_{ij}^P = \mu_j^Z \ \frac{ \phi_{ij} \cdot P_i }{ k_j^Z + \sum_{i}(\phi_{ij} \cdot P_i) } \ Z_j
\end{equation}
where $\mu_j^Z$ is the size-dependent maximum ingestion rate, $k_j^Z$ is the prey half-saturation level and $\phi_{ij}$ is the relative preference of $Z_j$ for prey type $P_i$.\\

Prey preference is assumed to vary with phytoplankton size $size_i^{P}$ in a log-Gaussian distribution around an optimal prey size for each grazer $size_j^{opt}$.

\begin{equation}
    \phi_{ij} = exp \left[ -\left( \ \frac{ log_{10}(size_i^{P}) - log_{10}(size_j^{opt}) }{ \Delta size^{P} } \right) \right]
\end{equation}
Where $\Delta size^{P}$ is the prey size tolerance parameter, with units of \unit{log_{10}(\mu m)}, that controls the width of the Gaussian distribution.\\

Zooplankton growth is a product of total biomass grazed ($G^P$) and the gross growth efficiency (GGE) of zooplankton. The two parameters defining GGE in this model are absorption efficiency ($\beta$) and net production efficiency ($\epsilon$). Adsorption efficiency $\beta$ describes the fraction of $G^P$ which is absorbed in the gut, of which the fraction $\epsilon$ is actually assimilated into biomass (to $Z$: \ $\beta \epsilon$), while the rest is excreted as DIN (to $N$: \ $\beta (1-\epsilon)$). GGE specifically is the product of $\beta$ and $\epsilon$, for which values between 0.2 and 0.3 have been observed for a wide range of zooplankton  \citep{Straile1997GrossGroup}. The fraction of $G^P$ egested (e.g. as faecal pellets) is calculated via $1-\beta$, which in the current model setup is assumed to be exported quickly in the flow-through setting and lost from the system. 

Natural mortality of zooplankton is implemented in the model via the linear mortality rate $m^Z$. Linear mortality feeds into the pool of $D$. In addition, zooplankton experiences quadratic mortality according to the parameter $m^{Z2}$, describing higher-order mortality and predation of zooplankton that is removed from the system \citep{Edwards2000TheModels}. This closure term is quadratic, calculated with the implicit assumption that mortality of $Z_j$ is proportional to total zooplankton biomass $\sum_{j} Z_j$.

\subsubsection{Detritus}
Detritus concentration in the upper layer ($D$) is supplied by all mortality of phytoplankton, linear zooplankton mortality and zooplankton egestion (e.g. faecal pellets). The loss terms are remineralisation, zooplankton grazing, mixing and an additional sinking flux. 

Detritus is remineralised at a constant rate $m^D$. Similar to $P_i$ and $Z_j$, $D$ is affected by mixing through changes in MLD, described by the mixing coefficient $K$. In addition to $K$, detritus experiences losses due to gravitational sinking at a rate of $v^D$. This term is added to describe the fast export of larger detritus particles below the mixed layer. 

\clearpage
\subsubsection{Model equations}
The rates of change of the state variables are described by the following set of equations. For the definition of all symbols used here see Table \ref{appendix:table:usecase3symbols}.
See the previous two use case sections (A 1.1 and A 1.2) for a more detailed discussion of model structure and formulation. 

%%%%%%%%%%%%%%%%%%%%%%%%%%%%%%%%%%%%%%%%%%%%%%%%%%%%%%%%%%%%%%%%%

\begin{equation}
    \frac{d N}{d t} = 
    K (N^0 - N) % Nutrient mixing
    +  \beta(1 - \epsilon)\sum_{j} \sum_{i} G_{ij}^P % Unassimilated grazing by Z
    +  m^D \ D % Remineralisation of D
    - \sum_{i} ( \mu_i^{P} \  P_i) % Phytoplankton gains
\end{equation}

%PHYTOPLANKTON
\begin{equation}
    \frac{d P_i}{d t} =
    \mu_i^{P} \   P_i  % Phytoplankton gains
    - m^P  \ \mu_i^{\emptyset} \ P_i % Linear mortality
    - \sum_{j} G_{ij}^P % Z grazing
    - K \ P_i% Phytoplankton mixing
\end{equation}

%ZOOPLANKTON
\begin{equation}
    \frac{d Z_j}{d t} =
    \beta \ \epsilon  \sum_{i} G_{ij}^P % Assimilated grazing
    - m^{Z} \ Z_j  % Linear mortality
    - m^{Z2} \ Z_j \ \sum_{j} Z_j  % Quadratic mortality
    - K \ Z_j % Phytoplankton mixing
\end{equation}

%DETRITUS
\begin{equation}
    \frac{d D}{d t} = 
    \sum_{i}(m^P  \ \mu_i^{\emptyset} \  P_i) % Linear mortality
    + (1 - \beta) \sum_{j} \sum_{i} G_{ij}^P % Unassimilated grazing by Z
    - m^D \ D % Remineralisation of D
    - K \ D % Mixing of D
    - \frac{v^D}{H} \ D % Sinking of D
\end{equation}

\clearpage
%TABLES & FIGURES



\begin{table*}[t]

\caption{ Definition of symbols employed in use case 3 appendix. (\unit{\mu M \ N} = \unit{mmol \ Nitrogen \ m^{-3}}) }

\begin{tabular}{l l l}
Symbol & Meaning & Unit\\
\tophline
\tophline
State variables:\\
\middlehline
$N$ & concentration of nutrient in the upper mixed layer & \unit{\mu M \ N} \\
$P_i$ & concentration of phytoplankton size class $i$ in the upper mixed layer & \unit{\mu M \ N} \\
$Z_j$ & concentration of zooplankton size class $j$ in the upper mixed layer & \unit{\mu M \ N} \\
$D$ & concentration of detritus in the upper mixed layer & \unit{\mu M \ N} \\

Forcings:\\
\middlehline
$N^\emptyset$ & nutrient concentration right below mixed layer & \unit{\mu M \ N} \\
$H$ & depth of the upper mixed layer (MLD) & \unit{m} \\
$h^+$ & positive derivative of H(t) & \unit{m \ d^{−1}}  \\
$I$ & irradiance at the surface & \unit{\mu mol \ photons \ m^{-2} \ s^{-1}} \\
$T$ & temperature of the upper mixed layer & \unit{\degree C} \\



Processes:\\
\middlehline
$K$ & material exchange between mixed and bottom layer & \unit{m \ d^{-1}} \\
$\gamma_i^N$ & nutrient-dependence of the growth of phytoplankton size class $i$ & dimensionless\\
$\gamma_T$ & temperature dependency of phytoplankton growth & dimensionless \\
$\gamma_I$ & light limitation of phytoplankton growth &  dimensionless\\
$G_{ij}^P$ & total biomass of phytoplankton size class $j$ grazed by zooplankton size class $i$ & \unit{\mu M \ N} \\
$\phi_{ij}$ & Preference of grazer $Z_j$ for prey $P_i$ & dimensionless \\


Parameters: \\
\middlehline
$\kappa$ & constant mixing parameter & \unit{m \ d^{−1}}  \\
$v^D$ & additional sinking parameter & \unit{m \ d^{−1}}  \\
$\mu_i^{\emptyset}$ & size-dependent maximum growth rate of phytoplankton size class $i$ & \unit{d^{-1}} \\
$k_i^N$ & half-saturation constant for nutrient uptake of  phytoplankton size class $i$ & \unit{\mu M \ N} \\
$m^P$ & fraction of size-dependent $\mu_i^{\emptyset}$ that is phytoplankton mortality & dimensionless \\

$\mu_j^Z$ & maximum ingestion rate for zooplankton size class $j$ &  \unit{d^{-1}} \\
$k_j^Z$ & Prey half-saturation level for zooplankton size class $j$ & \unit{\mu M \ N} \\
$size_i^{P}$ & Individual size of phytoplankton (prey) size class $i$ & \unit{\mu m} \\
$size_j^{opt}$ & Optimal prey size for zooplankton size class $j$ & \unit{\mu m} \\
$\Delta size^{P}$ & Prey size tolerance for grazers & \unit{log_{10}(\mu m)} \\

$\beta$ & absorption efficiency of zooplankton, fraction of $G_{ij}^P$ &  dimensionless \\
$\epsilon$ & net production efficiency of zooplankton, fraction of $\beta$ & dimensionless \\

$m^{Z}$ & linear mortality of zooplankton & \unit{ d^{-1}} \\
$m^{Z2}$ & quadratic mortality of zooplankton, dependent on total $Z$ biomass & \unit{(\mu M \ N)^{-1} \ d^{-1}} \\

%\middlehline
%\bottomhline
\end{tabular}
\label{appendix:table:usecase3symbols}
%\belowtable{This is a test} % Table Footnotes
\end{table*}
%


\clearpage

% this is a custom function to be able to see references when rendering subfiles:
\biblio

\end{document}
