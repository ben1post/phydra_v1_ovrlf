\documentclass[journal abbreviation, manuscript]{copernicus}

\begin{document}
\introduction  %% \introduction[modified heading if necessary]

% introduction should be adundantly clear with no unecessary details!
ANdrew notes:
- P1: why phytoplankton important?
- P2: why people model phytoplankton? (what are the types of model we use? npzd, pft, size, darwin, etc.)
- P3: all these different models, what is the problem? [Motivation for a better toolkit]
- P4: Here's my solution! Here I present flexible model framework. explain basic structure and function
"In this paper, we (a) describe modeling approach, (b) apply modeling approach in 3 examples"

can also mention open source, but just high-level summaries, same on phydra details, no detail in intro

establish, why what you make is different from everything done before. why it is important! (2 sentences)

2nd Section:
Background, theoretical framework. Specifics here!


\subsection{Biogeochemical role of phytoplankton}
- ocean ecosystem, understanding important, global change (important part of climate models)

\subsection{From NPZD to DARWIN}
quote Gentleman history of marine ecosystem modeling


\subsection{Open science means open source}
- lots of models, in lots of (often old) programming languages, varying implementations
- all different, hidden (+very complex) codebase, not easy to adapt \& modify

- but development in data science & other computer science areas, towards reproducible tools, towards shared frameworks and capabilities (like Benoît said in his blog post)
- tools should foster scientific collaboration
"easy to use, open source, easy to modify, toolbox"
- python Pangeo software ecosystem (jupyter, xarray, zarr)

\subsubsection{Xarray-simlab: object-oriented modeling framework}
- Further detail on the xarray-simlab framework used for phydra!

\subsubsection{phydra package}
- phydra: easy to use, open source, easy to modify, toolbox

quick intro, and set aim!

Here, we demonstrate the usage of the phydra v1 package in a simple NPZD setting and to showcase the flexibility, in a more complex size based trophic structure model. Both in slab physics.
The package can be used for any model setting, so far only slab physics are hard-coded in package.
Ultility of phydra v1 is showcased in a parameter fitting for Use Case I
Flexibility is showcased in Use Case II, where physical setting is kept the same, but multiple FTs added.
Finally, the relevance of this work and potential next steps are discussed.

Here, we demonstrate the usage of the phydra v1 package in a simple NPZD setting and to showcase the flexibility, in a more complex size based trophic structure model. Both in slab physics.
//
The utility of the phydra package is here shown via three model implementations. Two models from literature are recreated, and the flexible nature of this package allows a combination of as a third example. 

\end{document}




