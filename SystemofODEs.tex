%% Copernicus Publications Manuscript Preparation Template for LaTeX Submissions
%% ---------------------------------
%% This template should be used for copernicus.cls
%% The class file and some style files are bundled in the Copernicus Latex Package, which can be downloaded from the different journal webpages.
%% For further assistance please contact Copernicus Publications at: production@copernicus.org
%% https://publications.copernicus.org/for_authors/manuscript_preparation.html


%% Please use the following documentclass and journal abbreviations for discussion papers and final revised papers.

%% 2-column papers and discussion papers
\documentclass[journal abbreviation, manuscript]{copernicus}



%% Journal abbreviations (please use the same for discussion papers and final revised papers)


% Advances in Geosciences (adgeo)
% Advances in Radio Science (ars)
% Advances in Science and Research (asr)
% Advances in Statistical Climatology, Meteorology and Oceanography (ascmo)
% Annales Geophysicae (angeo)
% Archives Animal Breeding (aab)
% ASTRA Proceedings (ap)
% Atmospheric Chemistry and Physics (acp)
% Atmospheric Measurement Techniques (amt)
% Biogeosciences (bg)
% Climate of the Past (cp)
% DEUQUA Special Publications (deuquasp)
% Drinking Water Engineering and Science (dwes)
% Earth Surface Dynamics (esurf)
% Earth System Dynamics (esd)
% Earth System Science Data (essd)
% E&G Quaternary Science Journal (egqsj)
% European Journal of Mineralogy (ejm)
% Fossil Record (fr)
% Geochronology (gchron)
% Geographica Helvetica (gh)
% Geoscience Communication (gc)
% Geoscientific Instrumentation, Methods and Data Systems (gi)
% Geoscientific Model Development (gmd)
% History of Geo- and Space Sciences (hgss)
% Hydrology and Earth System Sciences (hess)
% Journal of Micropalaeontology (jm)
% Journal of Sensors and Sensor Systems (jsss)
% Magnetic Resonance (mr)
% Mechanical Sciences (ms)
% Natural Hazards and Earth System Sciences (nhess)
% Nonlinear Processes in Geophysics (npg)
% Ocean Science (os)
% Primate Biology (pb)
% Proceedings of the International Association of Hydrological Sciences (piahs)
% Scientific Drilling (sd)
% SOIL (soil)
% Solid Earth (se)
% The Cryosphere (tc)
% Weather and Climate Dynamics (wcd)
% Web Ecology (we)
% Wind Energy Science (wes)


%% \usepackage commands included in the copernicus.cls:
%\usepackage[german, english]{babel}
%\usepackage{tabularx}
%\usepackage{cancel}
%\usepackage{multirow}
%\usepackage{supertabular}
%\usepackage{algorithmic}
%\usepackage{algorithm}
%\usepackage{amsthm}
%\usepackage{float}
%\usepackage{subfig}
%\usepackage{rotating}


\begin{document}

\title{phydra V1 - system of equations}


% \Author[affil]{given_name}{surname}

\Author[1]{Benjamin}{Post}
% \Author[1]{}{}
% \Author[2]{}{}

\affil[1]{Leibniz Centre for Tropical Marine Ecology (ZMT) GmbH, Fahrenheitstr. 6, 28359 Bremen, Germany}
% \affil[]{ADDRESS}

%% The [] brackets identify the author with the corresponding affiliation. 1, 2, 3, etc. should be inserted.

%% If an author is deceased, please add a further affiliation and mark the respective author name(s) with a dagger, e.g. "\Author[2,$\dag$]{Anton}{Aman}" with the affiliations "\affil[2]{University of ...}" and "\affil[$\dag$]{deceased, 1 July 2019}"


\correspondence{Benjamin Post (benjamin.post@leibniz-zmt.de)}

\runningtitle{TEXT}

\runningauthor{TEXT}





\received{}
\pubdiscuss{} %% only important for two-stage journals
\revised{}
\accepted{}
\published{}

%% These dates will be inserted by Copernicus Publications during the typesetting process.

\firstpage{1}

\maketitle



\begin{abstract}
I am using the Copernicus latex template, which is the template for submitting manuscripts to GMD (Geoscientific Model Development). Following, I will write up all equations and parameters used in my simplified EMPOWER model in the \textit{phydra} package.
\end{abstract}


%' \copyrightstatement{TEXT}


\introduction  %% \introduction[modified heading if necessary]
The current model setup is based on the EMPOWER model \citep{Anderson2015c}, with minor modifications to the presentation of the equations, but no changes to model mechanisms or parameters.
The physical environment of the model uses slab physics as presented in \citet{Evans1985ACycles}. The model ocean is built up of a biologically inert deep ocean below a well mixed layer of seasonally variable depth that contains the ecosystem. The ecosystem component is a traditional NPZD model, containing four state variables describing nutrients $N$, phytoplankton $P$, zooplankton $Z$ and detritus $D$.
%%% EQUATIONS
%
%%% Single-row equation
%
%\begin{equation}
%
%\end{equation}
%
%%% Multi-line equation
%
%\begin{align}
%& 3 + 5 = 8\\
%& 3 + 5 = 8\\
%& 3 + 5 = 8
%\end{align}
%

%NUTRIENT
\section{Model Equations}
The rates of change of the state variables are described by the following set of equations:

\begin{equation}
    \frac{\partial N}{\partial t} = 
    K (N_0 - N) % Nutrient mixing
    + \beta(1 - k_{NZ})(G_P + G_D) % Unassimilated grazing by Z
    + m_D \cdot D % Remineralisation of D
    - \mu_{max}(T) \  \gamma^{I} \ \gamma^{N} \cdot P % Phytoplankton gains
\end{equation}

%PHYTOPLANKTON
\begin{equation}
    \frac{\partial P}{\partial t} =
    \mu_{max}(T) \  \gamma^{I} \ \gamma^{N} \cdot P  % Phytoplankton gains
    - m_P \cdot P % Linear mortality
    - m_{P2} \cdot P^2 % Quadratic mortality
    - G_P % Z grazing
    - K \cdot P % Phytoplankton mixing
\end{equation}

%ZOOPLANKTON
\begin{equation}
    \frac{\partial Z}{\partial t} =
    \beta k_{NZ}(G_P + G_D) % Assimilated grazing
    - m_Z \cdot Z % Linear mortality
    - m_{Z2} \cdot Z^2 % Quadratic mortality
    - K \cdot Z % Zooplankton mixing
\end{equation}

%DETRITUS
\begin{equation}
    \frac{\partial D}{\partial t} = 
    m_P \cdot P % Linear mortality
    + m_{P2} \cdot P^2 % Quadratic mortality
    + m_Z \cdot Z % Linear mortality
    + (1 - \beta)(G_P + G_D) % Unassimilated grazing by Z
    - G_D % Z grazing on D
    - m_D \cdot D % Remineralisation of D
    - K \cdot D % Mixing of D
    - \frac{v}{H(t)} \cdot D % Sinking of D
\end{equation}

\begin{table*}[t]
\caption{Symbols used in model equations}
\begin{tabular}{c c c}
Symbol & Meaning & Unit\\
\tophline
$K$ & material exchange between mixed and bottom layer & $[d^{-1}]$ \\
$N_0$ & nutrient concentration right below mixed layer & $[µM N]$ \\
$\beta$ & absorption efficiency of zooplankton grazing &  \\
$k_{NZ}$ & net production efficiency of zooplankton grazing &  \\
$G_P$ & total biomass of phytoplankton grazed by zooplankton & $[µM N]$ \\
$G_D$ & total biomass of detritus grazed by zooplankton & $[µM N]$ \\
$m_D$ & remineralisation rate of detritus & $[d^{-1}]$ \\
$\mu_{max}(T)$ & temperature-dependent max. growth rate of phytoplankton & $[d^{-1}]$ \\
$\gamma_I$ & light limitation of phytoplankton growth &  \\
$\gamma_N$ & nutrient limitation of phytoplankton growth &  \\
$m_P$ & linear mortality of phytoplankton & $[d^{-1}]$ \\
$m_{P2}$ & quadratic mortality of phytoplankton & $[d^{-1}]$ \\
$m_Z$ & linear mortality of zooplankton & $[d^{-1}]$ \\
$m_{Z2}$ & quadratic mortality of zooplankton & $[d^{-1}]$ \\
$v$ & sinking rate of detritus & $[m \cdot d^{-1}]$\\
%\middlehline
%\bottomhline
\end{tabular}
%\belowtable{This is a test} % Table Footnotes
\end{table*}
\subsection{Mixing}

The mixing coefficient $K$ describes mixing across the bottom of the mixed layer.

\begin{equation}
    K = \frac{1}{H(t)} \cdot \left(h^{+}(t) + \kappa\right)
\end{equation}

Constant diffusive mixing is parameterized by $\kappa$, which is further modified by the changes in MLD. The MLD at a certain time point is $H(t)$ and the change in MLD is its derivative $\frac{d}{d t} H(t)$. The function $h^{+}(t)$ gives the effects of entrainment and detrainment due to the changes in MLD.

\begin{equation}
    h^{+}(t) = \max\left(0, \frac{d}{d t} H(t)\right)
\end{equation}

The derivative of MLD $\frac{d}{d t} H(t)$ is positive when the mixed layer deepens. Nutrients are entrained from below, while other components of the ecosystem (e.g. phytoplankton) are diluted. When the mixed layer shallows, $h^{+}(t)$ does not modify $K$ (i.e. returns 0 instead of a negative value). It is assumed that detrainment of mass and the increase in concentration due to the reduced volume of the mixed layer are balanced.

$K$ affects all state variables in the upper mixed layer. In addition to $K$, detritus experiences losses due to gravitational sinking at a rate of $v$. This term is added to describe the fast export of larger detritus particles below the mixed layer. 

\subsection{Phytoplankton}
Phytoplankton biomass $P$ increases through temperature-dependent, light- \& nutrient-limited growth. Temperature dependence of the maximum growth rate $\mu_{max}(T)$ is calculated from the Eppley curve \citep{Eppley1972TemperatureSea}.

\begin{equation}
    \mu_{max}(T) = V^{max}_P = V^{max}_P(0) \cdot 1.066^{T_{MLD}} \label{mumax}
\end{equation}

With the assumption of balanced growth the maximum growth rate $\mu_{max}$ is equal to the maximum photosynthetic rate $V^{max}_P$. The temperature forcing $T_{MLD}$ is the average temperature of the mixed layer.

The light limitation of phytoplankton growth is described by the Smith photosynthesis-irradiance (P-I) function:

\begin{equation}
    V_P = \frac{\alpha \cdot I \cdot V^{max}_P}{\sqrt{(V^{max}_P)^2 + \alpha^2 I^2}}
\end{equation}
where $V_P$ is the photosynthetic rate, $\alpha$ is the initial slope of the P-I curve, $I$ is irradiance and $V^{max}_P$ is the temperature-dependent maximum photosynthetic rate defined in equation \eqref{mumax}.


The irradiance forcing $I_0$ is a time averaged measurement of photosynthetically active radiation (PAR) at the surface. Attenuation of $I_0$ at depth $z$ in the mixed layer is calculated according to the Lambert-Beer equation:

\begin{equation}
    I(z) = I_0 \cdot e^{-k_{PAR} \cdot Z} \label{beer}
\end{equation}

The attenuation coefficient $k_{PAR}$ is the sum of the attenuation coefficient of seawater $k_w$ and that of chlorophyll $k_c$, which is multiplied by phytoplankton biomass $P$:

\begin{equation}
    k_{PAR} = k_w + k_c \cdot P
\end{equation}

Combining the equations and integrating across the mixed layer, the numerical solution for the integrated light-limiting term affecting phytoplankton growth is:

\begin{equation}
    \gamma^I = \frac{V^P_{max}}{k_{PAR} \cdot H(t)} \ln{ \left( \frac{ x_0+\sqrt{(V^{max}_P)^2+x_0^2} }{ x_z+\sqrt{(V^{max}_P)^2+x_z^2} } \right)}
\end{equation}
where $x_0$ = $\alpha \cdot I_0$ and $x_z$ = $\alpha \cdot I(H(t))$, with $I(z)$ calculated according to equation \eqref{beer}.

Nutrient limitation of phytoplankton growth $\gamma^N$ is described by the Michaelis-Menten (or Monod) equation.

\begin{equation}
    \gamma^N = \frac{N}{k_N + N}
\end{equation}

where $k_N$ is the half-saturation constant. $N$ is nutrient concentration, in this case dissolved inorganic nitrogen (DIN).

Non-grazing mortality of phytoplankton is described by both a linear $m_P$ and a quadratic factor $m_{P2}$. The former accounts for natural mortality and excretion. Quadratic mortality describes density-dependent loss processes, which can be caused by viral infection. All non-grazing mortality terms feed into the detritus pool

\subsection{Zooplankton}
Grazing by zooplankton occurs on both phytoplankton and detritus. The grazing function is a Holling Type 3 grazing response as presented in \citet{Anderson2015c}.

\begin{equation}
    G_P = I_{max} \left( \frac{ \hat{\phi}_P P}{k_Z^2 + \hat{\phi}_D D +\hat{\phi}_P P}  \right) Z
\end{equation}
where $\hat{\phi}_P$ = $\phi_P \cdot P$, $\hat{\phi}_D$ = $\phi_D \cdot D$.

This formulation describes the total biomass of phytoplankton that is grazed $G_P$. Parameter $I_{max}$ is the maximum specific ingestion rate for a food source. Here the value is the same for both phytoplankton and detritus. The density dependent feeding coefficients $\hat{\phi}$ are calculated from multiplying the feeding parameter $\phi$ by prey biomass. The actual relative feeding preference can be calculated from the ration between $\hat{\phi}_D$ and $\hat{\phi}_P$. The arbitrary parameter $k_Z$ determines the half-saturation constants $k$ for grazing on a specific prey based on the choice of $\phi$, with the relationship $k$ = $\sqrt{\frac{k^2_Z }{ \phi}}$

For detritus the formula is the following:

\begin{equation}
    G_P = I_{max} \left( \frac{ \hat{\phi}_D D}{k_Z^2 + \hat{\phi}_D D +\hat{\phi}_P P}  \right) Z
\end{equation}

The sigmoidal response includes passive prey switching via an interference effect, where the increase in biomass of one prey slightly reduces the intake of other prey. The effect does not create sub-optimal feeding (i.e. an increase in biomass of a less common prey decreases total grazing), which can be observed in active switching grazing formulations as that used by \citet{Fasham1990a}.

Zooplankton ingestion of prey does not directly convert to biomass gained however. The total biomass grazed $G$ is split three ways between growth, excretion of dissolved nutrients and egestion of faecal matter \& particles. The parameters describing the contribution of each process are the absorption efficiency $\beta$ and net production efficiency $k_{NZ}$. Gross growth efficiency is the product of these two factors.  

Similar to phytoplankton mortality, the linear mortality factor $m_Z$ parameterizes natural mortality and excretion and feeds into the pool of detritus. The quadratic factor $m_{Z2}$ describes higher order predation on zooplankton and is removed from the system.

%%% TABLES
%%%
%%% The different columns must be separated with a & command and should
%%% end with \\ to identify the column brake.
%
%%% ONE-COLUMN TABLE
%
%%t
%\begin{table}[t]
%\caption{TEXT}
%\begin{tabular}{column = lcr}
%\tophline
%
%\middlehline
%
%\bottomhline
%\end{tabular}
%\belowtable{} % Table Footnotes
%\end{table}
%
%%% TWO-COLUMN TABLE
%
%%t
%\begin{table*}[t]
%\caption{TEXT}
%\begin{tabular}{column = lcr}
%\tophline
%
%\middlehline
%
%\bottomhline
%\end{tabular}
%\belowtable{} % Table Footnotes
%\end{table*}
%
%%% LANDSCAPE TABLE
%
%%t
%\begin{sidewaystable*}[t]
%\caption{TEXT}
%\begin{tabular}{column = lcr}
%\tophline
%
%\middlehline
%
%\bottomhline
%\end{tabular}
%\belowtable{} % Table Footnotes
%\end{sidewaystable*}
%
\section{Model Parameters}
The model parameters in table \ref{table:params} are taken directly from \citet{Anderson2015c}.
%%% TWO-COLUMN TABLE
%
%%t
\begin{table*}[t]
\caption{Model Parameters}
\begin{tabular}{c c c c c}
Parameter & Meaning & Unit & Temperate & Tropical \\
\tophline
$V^{max}_P$ & max photosynthesis rate & $[g \ C (g \ chl)^{−1} h^{−1}]$ & 2.5  &  \\
$\alpha$ & initial slope of P-I curve & $[g \ C (g \ chl)^{−1} h^{−1} (µE \ m^{-2} \ s^{-1})^{-1}]$ & 0.034 & \\
$k_N$ & half-sat. const: N uptake & $[µM \ N]$ & 0.85 & \\
$m_P$ & linear P mortality & $[d^{−1}]$ & 0.015 & \\
$m_{P2}$ & quadratic P mortality & $[(µM \ N)^{-1} d^{−1}]$ & 0.025 & \\
$I_{max}$ & Z max. ingestion rate & $[d^{−1}]$ & 1.0 & \\
$k_Z$ & Z half-saturation for intake & $[µM \ N]$ & 0.6 & \\
$\phi_P$ & grazing preference: P & 0.67 & & \\
$\phi_D$ & grazing preference: D & & 0.33 & \\
$\beta_Z$ & Z absorption efficiency & 0.69 & & \\
$k_{NZ}$ & Z net production efficiency & 0.75 & & \\
$m_Z$ & linear Z mortality  & $[d^{−1}]$ & 0.02 & \\
$m_{Z2}$ & quadratic Z mortality & $[(µM \ N)^{-1} d^{−1}]$ & 0.34 & \\
$v_D$ & D sinking rate & $[m \ d^{−1}]$ & 6.43 & \\
$m_D$ & D remineralisation rate & $[d^{−1}]$ & 0.06 & \\
$\kappa$ & constant mixing parameter & $[m \ d^{−1}]$ & 0.13 & \\
$\theta_{chla}$ & C to chlorophyll ratio & $[g \ g^{-1}]$ & 75 & \\
\middlehline

\bottomhline
\end{tabular}
\belowtable{The units for light harvesting are not final! I will most likely use mol photons, or whatever unit is SI and you all like.} % Table Footnotes
\label{table:params}
\end{table*}
%
\subsection{NOTES:}
Yes, the $V^{max}_P$ \& $\alpha$  units are pretty weird, and do require a unit conversion to $d^{-1}$ and the result of the light harvesting function needs to be divided by $\theta_{chla}$ to convert to the actual non-dimensional light limiting term. To me it seems like this is not well explained in the EMPOWER publication (still doesn't REALLY make sense to me), but I actually saw how they calculated it in the model code. I still need to work on how to present that understandably in the text before. 
%\conclusions  %% \conclusions[modified heading if necessary]
%TEXT

%% The following commands are for the statements about the availability of data sets and/or software code corresponding to the manuscript.
%% It is strongly recommended to make use of these sections in case data sets and/or software code have been part of your research the article is based on.

%\codeavailability{TEXT} %% use this section when having only software code available


%\dataavailability{TEXT} %% use this section when having only data sets available


%\codedataavailability{TEXT} %% use this section when having data sets and software code available


%\sampleavailability{TEXT} %% use this section when having geoscientific samples available


%\videosupplement{TEXT} %% use this section when having video supplements available


%\appendix
%\section{}    %% Appendix A

%\subsection{}     %% Appendix A1, A2, etc.


%\noappendix       %% use this to mark the end of the appendix section

%% Regarding figures and tables in appendices, the following two options are possible depending on your general handling of figures and tables in the manuscript environment:

%% Option 1: If you sorted all figures and tables into the sections of the text, please also sort the appendix figures and appendix tables into the respective appendix sections.
%% They will be correctly named automatically.

%% Option 2: If you put all figures after the reference list, please insert appendix tables and figures after the normal tables and figures.
%% To rename them correctly to A1, A2, etc., please add the following commands in front of them:

%\appendixfigures  %% needs to be added in front of appendix figures

%\appendixtables   %% needs to be added in front of appendix tables

%% Please add \clearpage between each table and/or figure. Further guidelines on figures and tables can be found below.



%\authorcontribution{TEXT} %% this section is mandatory

%\competinginterests{TEXT} %% this section is mandatory even if you declare that no competing interests are present

%\disclaimer{TEXT} %% optional section

%\begin{acknowledgements}
%TEXT
%\end{acknowledgements}




%% REFERENCES

%% The reference list is compiled as follows:

%\begin{thebibliography}{}

%\bibitem[AUTHOR(YEAR)]{LABEL1}
%REFERENCE 1

%\bibitem[AUTHOR(YEAR)]{LABEL2}
%REFERENCE 2

%\end{thebibliography}

%% Since the Copernicus LaTeX package includes the BibTeX style file copernicus.bst,
%% authors experienced with BibTeX only have to include the following two lines:
%%
\bibliographystyle{copernicus}
\bibliography{references.bib}
%%
%% URLs and DOIs can be entered in your BibTeX file as:
%%
%% URL = {http://www.xyz.org/~jones/idx_g.htm}
%% DOI = {10.5194/xyz}


%% LITERATURE CITATIONS
%%
%% command                        & example result
%% \citet{jones90}|               & Jones et al. (1990)
%% \citep{jones90}|               & (Jones et al., 1990)
%% \citep{jones90,jones93}|       & (Jones et al., 1990, 1993)
%% \citep[p.~32]{jones90}|        & (Jones et al., 1990, p.~32)
%% \citep[e.g.,][]{jones90}|      & (e.g., Jones et al., 1990)
%% \citep[e.g.,][p.~32]{jones90}| & (e.g., Jones et al., 1990, p.~32)
%% \citeauthor{jones90}|          & Jones et al.
%% \citeyear{jones90}|            & 1990



%% FIGURES

%% When figures and tables are placed at the end of the MS (article in one-column style), please add \clearpage
%% between bibliography and first table and/or figure as well as between each table and/or figure.


%% ONE-COLUMN FIGURES

%%f
%\begin{figure}[t]
%\includegraphics[width=8.3cm]{FILE NAME}
%\caption{TEXT}
%\end{figure}
%
%%% TWO-COLUMN FIGURES
%
%%f
%\begin{figure*}[t]
%\includegraphics[width=12cm]{FILE NAME}
%\caption{TEXT}
%\end{figure*}
%
%
%%% TABLES
%%%
%%% The different columns must be seperated with a & command and should
%%% end with \\ to identify the column brake.
%
%%% ONE-COLUMN TABLE
%
%%t
%\begin{table}[t]
%\caption{TEXT}
%\begin{tabular}{column = lcr}
%\tophline
%
%\middlehline
%
%\bottomhline
%\end{tabular}
%\belowtable{} % Table Footnotes
%\end{table}
%
%%% TWO-COLUMN TABLE
%
%%t
%\begin{table*}[t]
%\caption{TEXT}
%\begin{tabular}{column = lcr}
%\tophline
%
%\middlehline
%
%\bottomhline
%\end{tabular}
%\belowtable{} % Table Footnotes
%\end{table*}
%
%%% LANDSCAPE TABLE
%
%%t
%\begin{sidewaystable*}[t]
%\caption{TEXT}
%\begin{tabular}{column = lcr}
%\tophline
%
%\middlehline
%
%\bottomhline
%\end{tabular}
%\belowtable{} % Table Footnotes
%\end{sidewaystable*}
%
%
%%% MATHEMATICAL EXPRESSIONS
%
%%% All papers typeset by Copernicus Publications follow the math typesetting regulations
%%% given by the IUPAC Green Book (IUPAC: Quantities, Units and Symbols in Physical Chemistry,
%%% 2nd Edn., Blackwell Science, available at: http://old.iupac.org/publications/books/gbook/green_book_2ed.pdf, 1993).
%%%
%%% Physical quantities/variables are typeset in italic font (t for time, T for Temperature)
%%% Indices which are not defined are typeset in italic font (x, y, z, a, b, c)
%%% Items/objects which are defined are typeset in roman font (Car A, Car B)
%%% Descriptions/specifications which are defined by itself are typeset in roman font (abs, rel, ref, tot, net, ice)
%%% Abbreviations from 2 letters are typeset in roman font (RH, LAI)
%%% Vectors are identified in bold italic font using \vec{x}
%%% Matrices are identified in bold roman font
%%% Multiplication signs are typeset using the LaTeX commands \times (for vector products, grids, and exponential notations) or \cdot
%%% The character * should not be applied as mutliplication sign
%
%
%%% EQUATIONS
%
%%% Single-row equation
%
%\begin{equation}
%
%\end{equation}
%
%%% Multiline equation
%
%\begin{align}
%& 3 + 5 = 8\\
%& 3 + 5 = 8\\
%& 3 + 5 = 8
%\end{align}
%
%
%%% MATRICES
%
%\begin{matrix}
%x & y & z\\
%x & y & z\\
%x & y & z\\
%\end{matrix}
%
%
%%% ALGORITHM
%
%\begin{algorithm}
%\caption{...}
%\label{a1}
%\begin{algorithmic}
%...
%\end{algorithmic}
%\end{algorithm}
%
%
%%% CHEMICAL FORMULAS AND REACTIONS
%
%%% For formulas embedded in the text, please use \chem{}
%
%%% The reaction environment creates labels including the letter R, i.e. (R1), (R2), etc.
%
%\begin{reaction}
%%% \rightarrow should be used for normal (one-way) chemical reactions
%%% \rightleftharpoons should be used for equilibria
%%% \leftrightarrow should be used for resonance structures
%\end{reaction}
%
%
%%% PHYSICAL UNITS
%%%
%%% Please use \unit{} and apply the exponential notation


\end{document}
