%%% TABLES %%%%

\begin{table*}[t]
\caption{Allometric parameterisations and empirical parameter values employed in use case 2, adapted from \citet{Banas2011b}}
\begin{tabular}{l l l l l}
Empirical fit & Applicability & Source \\
\tophline
$\mu^i_{0} = (2.6 \ d^{-1}) \left( \frac{size^i_{P}}{1\mu m} \right)^{-0.45}$ & Phytoplankton 1-100 ESD \unit{\mu m} & Tang(1995) \\
$k^i_N = (0.1 \ \unit{\mu M \ N})\left( \frac{size^i_{P}}{1\mu m} \right)$ & Phytoplankton 1-100 ESD \unit{\mu m} & Eppley et al. (1969) \\

$I^j_0 = (26 \ d^{-1})\left( \frac{size^i_{P}}{1\mu m} \right)^{-0.4}$ & Flagellates, dinoflagellates, ciliates, copepods & Hansen et al. (1997) \\

$k^j_Z = 3 \ \unit{\mu M \ N} $ & Flagellates, dinoflagellates, ciliates, copepods & Hansen et al. (1997) \\

$size^j_{opt} = (0.65 \ \unit{\mu m})\left( \frac{size^i_{P}}{1\mu m} \right)^{0.56}$ & Flagellates, dinoflagellates, ciliates, copepods & Hansen et al. (1994) \\
$\Delta size_{P} = 0.25 $ & Ciliates, nauplii, copepodites & Hansen et al. (1994)  \\
\middlehline

\bottomhline
\end{tabular}
\belowtable{TODO: the sources need to be added properly! for now just text..} % Table Footnotes
\label{appendix:table:usecase2parameters}
\end{table*}




%%%%%


\subsubsection{Global nutrient, light and temperature climatologies as slab model forcing}
In line with the concept of phydra as a tool for rapid prototyping of marine ecosystem models, we compiled a set of global climatological forcings for slab models. These forcings are derived from World Ocean Atlas (WOA) 2018 data and a recent global MLD climatology kindly provided by Clément de Boyer Montégut, and Moderate Resolution Imaging Spectroradiometer (MODIS-aqua) satellite data for the time period 2002–2019.

WOA data provides objectively analysed climatological mean depth profiles of nutrients (nitrate, phosphate \& silicate) and temperature on 1 \unit{°} longitude/latitude grid \cite{Garcia2019WORLDSilicate}. The values have been interpolated from data collected in the World Ocean Database and provide an empirical estimate of the biogeochemical conditions in areas of the global ocean throughout the year. WOA 2018 data is maintained and provided by NOAA (\url{https://www.nodc.noaa.gov/OC5/woa18/woa18data.html, accessed October 2019}).\\

The MLD climatology is an updated version of the original climatology presented in  \citet{deBoyerMontegut2004MixedClimatology}, collecting data up until 2014 and with a modified criterion for MLD. The analysis of profile data combines a fixed threshold criterion for temperature (0.2 \unit{°C}) and a variable threshold criterion in density (equivalent to a 0.2 \unit{°C} decrease). MLD is diagnosed as the minimum calculated depth of both criterion for each station. This ensures that both temperature and salinity are homogeneous within the mixed layer, and compensated or barrier layers do not skew the values of MLD. This combined MLD criterion corresponds to a proxy of overturning extent depth over a few days, which lends this MLD value very well as the forcing that drives the upwelling of deeper nutrients into the upper mixed layer in slab physics (Clément de Boyer Montégut, personal communication). Spatial resolution of the climatology is on a 2 \unit{°} longitude/latitude grid.\\

In addition to the oceanographic and biogeochemical parameters, the provided set of global forcings includes a global climatology of irradiance from satellite data. The NASA satellite MODIS-aqua provides the most up-to-date global climatologies of photosynthetically active radiation (PAR) \cite{MODIS-Aqua2018NASAGroup}. 
\\
With this set of forcings we can experimentally run a slab model built in phydra in (almost) any location of the ocean. MLD can be used as an empirical forcing for mixing in our slab models. Average nutrient concentration below the mixed layer, extracted from the WOA 2018 climatologies, can be used to provide an estimate of nutrient supply at certain locations throughout the year. The deep nutrient climatology averages WOA 2018 data 5 \unit{m} below closest available climatological MLD value. The final 1 \unit{°} gridded data product can be accessed via Github [provide link] and is easily integrated with models built using the phydra package. 
The monthly climatologies are spatially averaged over a selected location (see Figure \ref{phydraforcing} a) and interpolated to obtain daily values (see Figure \ref{phydraforcing} b).






\subsection{Xarray-Simlab: object-oriented modelling framework}
%!
\textit{For now this subsection just includes my bullet points, Benoît Bovy has promised to write it.}
%!

A key point is in the name, xarray-simlab builds a flexible model structure on top xarray -> for labelled multidimensional datasets in python.

This means models in xarray-simlab natively use xarray as the basis for the data structure. A model with the proper parameterisation creates an xarray-dataset, which stores the output after model is run.

This 

- Further detail on the xarray-simlab framework used for phydra!

"xarray-simlab is a Python library that provides both a generic framework for building computational models in a modular fashion and a xarray extension for setting and running simulations using the xarray.Dataset structure. It is designed for fast, interactive and exploratory modelling."


specific version used here: v0.4.1 \cite{benoit_bovy_2020_3755979}


The xarray-simlab framework is built on a very few concepts that allow great flexibility in model customisation:

- models
- processes
- variables

\subsubsection{Models, processes and variables}
in xarray-simlab, (almost) everything is a process.
- in the following sections, introduce the specific implementation of xarray-simlab processes within phydra, what is actually implemented in v1 and how it can be modified and used.

models

processes

Just describe this in text...!

variables
- foreign variables
- group variables
- on-demand variables
- index variables
- object variables


\subsection{GEKKO Optimization suite: compiled solver back end}

Automatic differentiation provides the necessary gradients, accurate to machine precision, without extra work from the user.
that allows more efficient computation of larger models, a modular mathematical syntax from which to build our model,
and a powerful framework for model optimisation.


Model forcings and data used in model optimisation are represented by GEKKO parameters that are supplied to the model discretised at the same time-step that the model is solved.

The GEKKO package 

"GEKKO is an object-oriented Python library that offers model construction, analysis tools, and visualisation of simulation and optimisation."

"The GEKKO Python package solves large-scale mixed-integer and differential algebraic equations with nonlinear programming solvers. Modes of operation include machine learning, data reconciliation, real-time optimization, dynamic simulation, and nonlinear model predictive control." \cite{Beal2018GekkoSuite}

"GEKKO is not only an algebraic modeling language (AML) for posing optimization problems in simple object-oriented equation-based models to interface with powerful built-in optimization solvers but is also a package with the built-in ability to run model predictive control, dynamic parameter estimation, real-time optimization, and parameter update for dynamic models on real-time applications. The"

"Algebraic modeling languages (AML) facilitate the interface between advanced solvers and
human users. High-end, off-the-shelf gradient-based solvers require extensive information about the problem, including variable bounds, constraint functions and bounds, objective functions, and first and second derivatives of the functions, all in consistent array format. AMLs simplify the process by allowing the model to be written in a simple, intuitive format."

"GEKKO fills the role of a typical AML, but extends its capabilities to specialize in dynamic
optimization applications. As an AML, GEKKO provides a user-friendly, object-oriented Python interface to develop models and optimization solutions. Python is a free and open-source language that is flexible, popular, and powerful. IEEE Spectrum ranked Python the #1 programming language in 2017. Being a Python package allows GEKKO to easily interact with other popular scientific and numerical packages. Further, this enables GEKKO to connect to any real system that can be accessed through Python. Since Python is designed for readability and ease rather than speed, the Python GEKKO model is
converted to a low-level representation in the Fortran back-end for speed in function calls. Automatic differentiation provides the necessary gradients, accurate to machine precision, without extra work from the user. GEKKO then interacts with the built-in open-source, commercial, and custom large-scale solvers for linear, quadratic, nonlinear, and mixed integer programming (LP, QP, NLP, MILP, and MINLP) in the back-end. Optimization results are loaded back to Python for easy access and further analysis or manipulation."

specific version used here: v0.2.7

\subsubsection{Gekko variable types}
there are more, Gekko is much more complex than our use case, but the ones used in this implementation are
- Parameters. 
- Intermediates. 
- State Variables.  
- Equations.
additionally some functions like summing of terms (sum(x)) or calculating the Euler exponent (exp(x)) have specific implementations that are part of the Gekko library.




% FROM DIMENSIONALITY SECTION:
- Dimensionality is a very important fact of building models, and xarray-simlab provides a simple, but sometimes tricky interface for managing dimensionality, so make sure to explain it relatively well here, as well as the implementations.
- I can cite https://www.biogeosciences.net/17/609/2020/ to explain why dimensionality is an important consideration in phytoplankton models.




Growth & $\mu_{\emptyset}$ & maximum phytoplankton growth rate  & \unit{d^{-1}} & & \\
& $k_N$ & half-sat. const: N uptake & \unit{µM \ N} & 0.85 & \\
& $m_P$ & linear P mortality & \unit{d^{−1}} & 0.015 & \\
& $m_{P2}$ & quadratic P mortality & \unit{(µM \ N)^{-1} d^{−1}} & 0.025 & \\
& $I_{max}$ & Z max. ingestion rate & \unit{d^{−1}} & 1.0 & \\
& $k_Z$ & Z half-saturation for intake & \unit{µM \ N} & 0.6 & \\
& $\phi_P$ & grazing preference: P & & 0.67 & \\
& $\phi_D$ & grazing preference: D & & 0.33 & \\
& $\beta_Z$ & Z absorption efficiency & & 0.69 & \\
& $k_{NZ}$ & Z net production efficiency & & 0.75 &  \\
& $m_Z$ & linear Z mortality  & \unit{d^{−1}} & 0.02 & \\
& $m_{Z2}$ & quadratic Z mortality & \unit{(µM \ N)^{-1} d^{−1}} & 0.34 & \\
& $v_D$ & D sinking rate & \unit{m \ d^{−1}} & 6.43 & \\
& $m_D$ & D remineralisation rate & \unit{d^{−1}} & 0.06 & \\
& $\kappa$ & constant mixing parameter & \unit{m \ d^{−1}} & 0.13 & \\





\subsubsection{Environments} \label{Section:PhysicalEnvironment}

A Phydra Environment defines the structure of the ecosystem and provides external inputs (i.e. forcings) that can be accessed by all processes linked to this environment. As a simple example we could take a chemostat model. The flow-through culture system is defined by a solution of nutrients that flows into a tank of fixed volume. Organisms within the tank grow, consume nutrients and are also flushed out of the system with the outflow of the medium. The Environment process in Phydra describing a chemostat would have the nutrient concentration in the medium, as well as the flow-rate of the system as required external forcing processes. The forcing input can be supplied from different Forcing processes, for example with a time-varying flow rate. Components, which define the state variables in our model, are grouped within our chemostat Environment and the Fluxes (e.g. nutrient uptake or mortality) affect Components within this Environment. We can add any number of Components from the Phydra library into our Environment and between the Components we could add any number of interactions.

Phydra as of now provides an easily modifiable base environment and implementations of a chemostat environment, as well as a slab-ocean environment. The first release of Phydra supports only single 0-dimensional implementations of these basic ecosystem models, but extending this to 2- and 3-dimensional flexibility is a high development priority. 

\subsubsection{Forcings} \label{Section:ForcingSection}

Forcings are defined as providing an external constant or time-varying parameter to the model. These parameters can be used in fluxes that reference the forcing type, but are most often used in conjunction with an environment that takes the specific type of forcing as an input. Included with this version of Phydra we have the necessary forcing types that have to be supplied to the provided environments. For a chemostat model, this is a source concentration of a component in the medium, as well as a flow rate of the system. The slab-ocean Environment interfaces with forcings describing the concentration of a component below the mixed layer, movement of the mixed layer depth (MLD), irradiance at surface and temperature within the mixed layer. There are basic methods supplied to create either constant forcing or values over time based on mathematical functions. Additionally a base forcing can be modified by a user to provide any external data, which can then be used by other processes in the library. The flexible class-based structure allows a process inheriting from the base MLD forcing process to be recognised as an MLD forcing in the fluxes and processes of our model.

\subsubsection{Components}

An ecosystem model tracks chemical compounds as well as organisms via state variables. These state variables can define completely different components of a model, or represent a functional group. Components within Phydra are defined at this higher level and can contain a single state variable or an array of state variables that share common fluxes with differing parameterisation. Each component is added to the model with a specific label that is used to reference this component in all processes affecting or dependent on it. The actual dimensions of a model component are initialized after passing a parameter at model setup. This has been designed for easily testing different levels of ecosystem complexity. As shown in the second use case, where this is useful is in setting up phytoplankton size classes in trait-based models. All phytoplankton state variables grow on a common nutrient, but uptake parameters are related to allometries of cell size. A phytoplankton component could be initialized, with a certain size range and number of state variables. Within the allometric growth process parameters are dynamically initialized based on the size provided by the component. 
In addition to size, the component could be modified to include information on units or other specific parameters relevant to the model. The added flexible dimensionality of components was designed with the current issues in marine ecosystem model in mind. The effects of different levels of complexity, in the number and definition of phytoplankton functional types (PFT) for example, is not routinely tested in marine ecosystem models. Phydra provides a framework that allows for easy testing through flexible modification of such model complexity at model setup.



\subsubsection{Fluxes}
The previous building blocks of an environment, components and forcing create the structure of the model, but when solved at this stage there would be no meaningful simulation. In order to define exchanges of matter between components flux processes are added to our model instance, defining the affected components via labels at model setup.
There are multiple types of fluxes provided as base classes in the library. These base classes are defined by the type of interaction between components. Single fluxes provide loss or gain processes to a single variable, such as sinking or influx from outside the ecosystem. In order to simplify creating common forcing fluxes, a single flux process can take list of affected components as input at model setup. An example for more complex flux type are exchange fluxes, where a flux affects one component as a source and another as a sink. These can be set up with flexible dimensionality of components. Grazing fluxes are a slightly more complicated subclass of a grid-wise flux, as ingestion is usually normalized by total prey availability, so all forcing interactions need to be calculated in a dynamically generated matrix of all prey items consumed by a component.

The calculation of a primary producers growth rate is usually one of the more complex fluxes in a marine ecosystem, perhaps second to grazing formulations. In the most simple case the growth of phytoplankton in our slab model would be an exchange flux from the component that grows, dependent on a saturating Monod (or Michealis-Menten) function of a resource concentration. There is no distinction between the rate of nutrient uptake and the assimilation of these nutrients. We can add this simplified relationship to our model using the MonodGrowth process, which is applied in our second use case. At model setup the user only needs to provide the label of the resource that is consumed and the component that is growing, in addition to parameters such as the half-saturation constant for uptake, to have this growth function implemented.
More complex representations of growth in marine ecosystem models track additional factors limiting growth. In our first use case, phytoplankton growth is limited by nutrient uptake, according to Monod, as well as temperature and light-availability in the mixed layer. These additional growth limiting factors are included in the library as a MonodLightTempGrowth process, that inherits the forcing for average temperature in the mixed layer and PAR at surface from the slab environment.