\documentclass[template.tex]{subfiles}


\usepackage{comment}

\begin{document}

%' \copyrightstatement{TEXT}
\section{Use case model formulations}

\begin{comment}
Note: to allow for an intuitive depiction of multiple instances of each parameter and state variable, modifying subscripts in symbols are instead added as a superscript (e.g. $\gamma^{N}$ for the nutrient-limitation of phytoplankton growth) so that indices in the subscript signify the dimensionality of symbols (e.g. $\gamma^{N}_i$ for the term per individual phytoplankton state variable $i$). This finds no usage in use case 1, but in both 2 and 3.
Overall I attempted to make the presentation of symbols and equations coherent between use cases.
\end{comment}

In this appendix the model use cases are described mathematically, independent of their specific implementation in the phydra package. For parameter values used in model runs, please see the Tables \ref{Table:UseCase1Parameters,2,3} in Section \ref{Section:UseCases}.

\subsection{Use Case 1}

For the first use case we have implemented a traditional NPZD ecosystem model as presented in Figure \ref{Figure:phydraschematics_1} (a) with a nutrient $N$ (in this case nitrate), phytoplankton $P$, zooplankton $Z$ and detritus $D$ as state variables. The first use case model employs slab physics as presented in \citet{Evans1985ACycles}. The model ocean is built up of two layers. A biologically inert deep ocean is situated below a well mixed upper layer of variable depth that contains the ecosystem. The general model structure is adapted from the EMPOWER model presented by \citet{Anderson2015c} with simplifications to the treatment of light in the model and it's parameterisation. 
Modifications were aimed at simplifying the description of physical forcings and phytoplankton growth. The model is driven by empirical forcing describing the depth of the mixed layer ($H$), average temperature of the mixed layer ($T$), photosynthetically active radiation at the surface ($I$) and nutrient concentration in the deep layer ($N^\emptyset$)

\\

% modifications of EMPOWER:
%- light harvesting\\ 
%- light $PAR$ & nutrient $N^\emptyset$ forcing\\ 
%- temperature dependence of phytoplankton growth?.\\ 


\subsubsection{Mixing}

% Nutrient dynamics
The zero-dimensional physical slab setting describes two vertical layers of which the deeper layer supplies nutrients to the upper layer, whilst other components are mixed to the deep layer and lost from the system.
The magnitude of mixing is described by the coefficient $K$:

\begin{equation}
    K = \frac{h^{+} + \kappa}{H}
\end{equation}

Constant diffusive mixing is parameterized by $\kappa$. Variable mixing is a function of the change in MLD over time $h = \frac{d}{d t} H$. The derivative of MLD ($h$) is positive when the mixed layer deepens. The function $h^{+}$ defines the effects of entrainment and detrainment due to the changes in MLD as $h^{+} = \max(0, \ h)$. When the mixed layer shallows, $h^{+}$ does not modify $K$ (i.e. returns 0 instead of a negative value), based on the assumption that detrainment of mass and the increase in concentration due to the reduced volume of the mixed layer are balanced \citep{Evans1985ACycles}. \\

\subsubsection{Nutrients}
Dissolved inorganic nitrogen in the mixed layer ($N$) is supplied via mixing, zooplankton excretion and detritus remineralisation.
Nutrients are entrained from the bottom layer. Mixing of nutrients is a positive term adding to $N$ along the gradient between $N^\emptyset$ and $N$. The general direction of transport is from a nutrient-rich bottom layer to the upper layer supporting phytoplankton growth, which is the only loss term.

\subsubsection{Phytoplankton}
Phytoplankton biomass ($P$) increases through temperature-dependent, light- \& nutrient-limited growth. The growth rate ($\mu^{P}$) is the product of a maximum growth rate ($\mu^{\emptyset}$) and the growth-dependencies on temperature ($\gamma^{T}$), light ($\gamma^{I}$) and nutrients ($\gamma^{N}$): 

\begin{equation}
    \mu^{P} = \mu^{\emptyset} \ \gamma^{T} \ \gamma^{I} \ \gamma^{N}
\end{equation}

$T$ is the average temperature of the mixed layer in \unit{\degree C}, as supplied from model forcing. Temperature dependence of the growth rate ($\gamma^{T}$) is calculated via the Eppley curve \citep{Eppley1972TemperatureSea}.

\begin{equation}
    \gamma^{T} = \exp{(0.063 \ T)} \label{mumax}
\end{equation}

The light-limiting term $\gamma^{I}$ represents growth-dependence on total light ($I$) available to phytoplankton the upper mixed layer. We use a simplified form of Steele's formulation to describe light-limitation of phytoplankton growth in the mixed layer, as adapted from \citet{Acevedo-Trejos2016} and originally described in \citet{Steele1962EnvironmentalSea}.

\begin{equation}
    \gamma^{I} = \frac{1}{H} \int_{0}^{H}\left[ \frac{I(z)}{I^{opt}} \cdot \exp{\left( 1 - \frac{I(z)}{I^{opt}} \right) }  \right]dz \label{steele}
\end{equation}

Where $I^{opt}$ is the light level at which photosynthesis saturates and $I(z)$ is the irradiance at depth $z$.
The irradiance forcing $I$ is a temporally and spatially averaged monthly climatology of photosynthetically active radiation (PAR) at the surface. 

Attenuation of $I$ at depth $z$ in the mixed layer is calculated according to the Lambert-Beer equation:

\begin{equation}
    I(z) = I \ \exp{(-k^{PAR} \ z)} \label{beer}
\end{equation}

The attenuation coefficient $k^{PAR}$ is the sum of the attenuation coefficient of seawater $k^w$ and that of phytoplankton biomass $k^c$, which is multiplied by the current phytoplankton biomass $P$:

\begin{equation}
    k^{PAR} = k^w + k^c \cdot P
\end{equation}

Combining the equations and integrating across the mixed layer, the numerical solution for the integrated light-limiting term affecting phytoplankton growth is calculated. Integrated values within the mixed layer larger than the optimal irradiance will limit growth, to model effects of photo-inhibition \citep{Steele1962EnvironmentalSea}.
This simple implementation uses only one parameter to describe light limitation of phytoplankton growth, which is $I^{opt}$. This is a highly simplified treatment of light in a slab model. A more realistic representation of light-limited growth would include chlorophyll-to-biomass ratios and related parameters and functions. See \citet{Anderson2015c} for an insightful discussion of light-limitation in slab models.\\

Nutrient limitation of phytoplankton growth $\gamma^N$ is described by the Michaelis-Menten (or Monod) equation.

\begin{equation}
    \gamma^N = \frac{N}{k^N + N}
\end{equation}

where $k^N$ is the half-saturation constant for nutrient uptake. $N$ is the ambient nutrient concentration, in this case of dissolved inorganic nitrogen (DIN). In this simple model there is no distinction between nutrient uptake and assimilation of nutrient via growth.\\

Non-grazing mortality of phytoplankton is described by both a linear $m^P$ and a quadratic factor $m^{P2}$ \citep{Yool2011Medusa-1.0:Domain}. The former accounts for natural mortality and excretion. Quadratic mortality describes density-dependent loss processes, which can be caused by viral infection. All non-grazing loss terms feed into the detritus pool.

\subsubsection{Zooplankton}
Grazing by zooplankton occurs on both phytoplankton and detritus. The grazing function is a Holling Type 3 grazing response as presented in \citet{Anderson2015c}:

\begin{equation}
    G^P = \mu^Z \left( \frac{ \hat{\phi}^P P}{(k^Z)^2 + \hat{\phi}^D D +\hat{\phi}^P P}  \right) Z
\end{equation}
where $\hat{\phi}^P$ = $\phi^P \ P$, $\hat{\phi}^D$ = $\phi^D \ D$.

This formulation describes the total biomass of phytoplankton that is grazed $G^P$. Parameter $\mu^Z$ is the maximum ingestion rate for a food source, in this case both phytoplankton and detritus. 
The grazing preference parameters $\phi^P$ and $\phi^D$ do not represent a discrete fraction of the amount grazed in the diet relative to the environment. Instead, this amount is represented by the ratio of $\hat{\phi}^P$ and $\hat{\phi}^D$. 
The half-saturation constant for grazing $k^Z$ is similarly an arbitrary parameter, that scales the density-dependent half-saturation constant $k^P$ for grazing on phytoplankton based on the choice of $\phi^P$, with the relationship $k^P$ = $\sqrt{\frac{(k^Z)^2 }{ \phi^P}}$.


Similarly the detritus grazing flux is defined as:

\begin{equation}
    G^D = \mu^Z \left( \frac{ \hat{\phi}^D D}{(k^Z)^2 + \hat{\phi}^D D +\hat{\phi}^P P}  \right) Z
\end{equation}

This sigmoidal response includes passive prey switching via an interference effect, where the increase in biomass of one prey slightly reduces the intake of other prey. In contrast to other grazing formulations \citep[e.g.][]{Fasham1990a}, the prey switching mechanism does not create sub-optimal feeding, where an increase in biomass of less common prey can decreases the total grazing flux \citep{Gentleman2003a}.

Zooplankton ingestion of prey does not directly convert to biomass gained however. The total biomass grazed ($G^P + G^D$) is split three ways between zooplankton growth (to $Z$), excretion of dissolved nutrients (to $N$) and egestion of faecal matter \& particles (to $D$). Zooplankton growth is a product of total biomass grazed ($G^P$) and the gross growth efficiency (GGE) of zooplankton. The two parameters defining GGE in this model are absorption efficiency ($\beta$) and net production efficiency ($\epsilon$). Adsorption efficiency $\beta$ describes the fraction of $G^P$ which is absorbed in the gut, of which the fraction $\epsilon$ is actually assimilated into biomass (to $Z$: \ $\beta \epsilon$), while the rest is excreted as DIN (to $N$: \ $\beta (1-\epsilon)$). GGE specifically is the product of $\epsilon$ and $\beta$, for which values between 0.2 and 0.3 have been observed for a wide range of zooplankton \citep{Straile1997GrossGroup}. The fraction of $G^P$ egested to $D$ (e.g. as faecal pellets) is calculated via $1-\beta$. 

Similar to phytoplankton mortality, the linear mortality factor $m^Z$ parameterizes natural mortality and excretion and feeds into the pool of detritus. The quadratic factor $m^{Z2}$ describes higher order predation on zooplankton and is removed from the system. 

\subsection{Detritus}
Detritus concentration in the upper layer ($D$) is supplied by all mortality of phytoplankton, linear zooplankton mortality and zooplankton egestion (e.g. faecal pellets). The loss terms are remineralisation, zooplankton grazing, mixing and an additional sinking flux. 

Detritus is remineralised at a constant rate $m^D$. Similar to $P$ and $Z$, $D$ is affected by mixing through changes in MLD, described by the mixing coefficient $K$. In addition to $K$, detritus experiences losses due to gravitational sinking at a rate of $v^D$. This term is added to describe the fast export of larger detritus particles below the mixed layer. 

\clearpage
\subsubsection{Full model equations}
The rates of change of the state variables are described by the following set of equations. For the definition of all symbols used here see Table \ref{appendix:table:usecase1symbols}. See \citet{Anderson2015c} for a more detailed discussion of model structure and formulation.

\begin{equation}
    \frac{d N}{d t} = 
    K (N^\emptyset - N) % Nutrient mixing
    + \beta(1 - \epsilon)(G^P + G^D) % Unassimilated grazing by Z
    + m^D \ D % Remineralisation of D
    - \mu^{P} \ P % Phytoplankton gains
\end{equation}

%PHYTOPLANKTON
\begin{equation}
    \frac{d P}{d t} =
    \mu^{P} \ P  % Phytoplankton gains
    - m^P \ P % Linear mortality
    - m^{P2} \ (P)^2 % Quadratic mortality
    - G^P % Z grazing
    - K \ P % Phytoplankton mixing
\end{equation}

%ZOOPLANKTON
\begin{equation}
    \frac{d Z}{d t} =
    \beta \ \epsilon(G^P + G^D) % Assimilated grazing
    - m^Z \ Z % Linear mortality
    - m^{Z2} \ (Z)^2 % Quadratic mortality
    - K \ Z % Zooplankton mixing
\end{equation}

%DETRITUS
\begin{equation}
    \frac{d D}{d t} = 
    m^P \ P % Linear mortality
    + m^{P2} \ (P)^2 % Quadratic mortality
    + m^Z \ Z % Linear mortality
    + (1 - \beta)(G^P + G^D) % Unassimilated grazing by Z
    - G^D % Z grazing on D
    - m^D \ D % Remineralisation of D
    - K \ D % Mixing of D
    - \frac{v^D}{H} \ D % Sinking of D
\end{equation}



\clearpage
%TABLES & FIGURES



\begin{table*}[t]

\caption{ Definition of symbols employed in use case 1 appendix. (\unit{\mu M \ N} = \unit{mmol \ Nitrogen \ m^{-3}}) }

\begin{tabular}{l l l}
Symbol & Meaning & Unit\\
\tophline
\tophline
State variables:\\
\middlehline
$N$ & concentration of nutrient in the upper mixed layer & \unit{\mu M \ N} \\
$P$ & concentration of phytoplankton biomass in the upper mixed layer & \unit{\mu M \ N} \\
$Z$ & concentration of zooplankton biomass in the upper mixed layer & \unit{\mu M \ N} \\
$D$ & concentration of detritus in the upper mixed layer & \unit{\mu M \ N} \\
\\

Forcings:\\
\middlehline
$N^\emptyset$ & nutrient concentration right below mixed layer & \unit{\mu M \ N} \\
$H$ & depth of the upper mixed layer (MLD) & \unit{m} \\
$h^+$ & positive derivative of H(t) & \unit{m \ d^{−1}}  \\
$I$ & irradiance at the surface & \unit{\mu mol \ photons \ m^{-2} \ s^{-1}} \\
$T$ & temperature of the upper mixed layer & \unit{\degree C} \\
\\
Processes:\\
\middlehline
$K$ & material exchange between mixed and bottom layer & \unit{m \ d^{-1}} \\
$\gamma_T$ & temperature dependency of phytoplankton growth & dimensionless \\
$\gamma_I$ & light limitation of phytoplankton growth &  dimensionless\\
$\gamma_N$ & nutrient limitation of phytoplankton growth & dimensionless \\
$G_P$ & total biomass of phytoplankton grazed by zooplankton & \unit{\mu M \ N} \\
$G_D$ & total biomass of detritus grazed by zooplankton & \unit{\mu M \ N} \\
\\
Parameters: \\
\middlehline
$\kappa$ & constant mixing parameter & \unit{m \ d^{−1}}  \\
$v^D$ & additional sinking parameter & \unit{m \ d^{−1}}  \\
$\mu^\emptyset$ & maximum phytoplankton growth rate & \unit{d^{−1}}  \\
$I^{opt}$ & constant mixing parameter & \unit{\mu mol \ photons \ m^{-2} \ s^{-1}}  \\
$k^w$ & light attenuation constant of sea water & \unit{m^{−1}}  \\
$k^c$ & light attenuation constant of phytoplankton biomass & \unit{m^{−1}}  \\
$k^N$ & half-saturation constant for nutrient uptake & \unit{\mu M \ N}  \\
$\mu^Z$ & maximum ingestion rate of zooplankton & \unit{d^{−1}}  \\
$\phi_P$ & zooplankton grazing preference for phytoplankton & dimensionless \\
$\phi_D$ & zooplankton grazing preference for zooplankton & dimensionless \\
$k_Z$ & half-saturation constant for zooplankton intake & \unit{\mu M \ N} \\
$\beta$ & absorption efficiency of zooplankton grazing &  dimensionless \\
$\epsilon$ & net production efficiency of zooplankton grazing & dimensionless \\
$m^D$ & remineralisation rate of detritus & \unit{d^{-1}} \\
$m^P$ & linear mortality of phytoplankton & \unit{d^{-1}} \\
$m^{P2}$ & quadratic mortality of phytoplankton & \unit{(\mu M \ N)^{-1} \ d^{-1}} \\
$m^Z$ & linear mortality of zooplankton & \unit{d^{-1}} \\
$m^{Z2}$ & quadratic mortality of zooplankton & \unit{(\mu M \ N)^{-1} \ d^{-1}} \\
%\middlehline
%\bottomhline
\end{tabular}
\label{appendix:table:usecase1symbols}
%\belowtable{This is a test} % Table Footnotes
\end{table*}



\clearpage

% this is a custom function to be able to see references when rendering subfiles:
\biblio

\end{document}
