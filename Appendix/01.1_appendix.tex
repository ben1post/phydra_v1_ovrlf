\documentclass[template.tex]{subfiles}


\begin{document}

%' \copyrightstatement{TEXT}
\section{Use case model formulations}
%%% EQUATIONS
%
%%% Single-row equation
%
%\begin{equation}
%
%\end{equation}
%
%%% Multi-line equation
%
%\begin{align}
%& 3 + 5 = 8\\
%& 3 + 5 = 8\\
%& 3 + 5 = 8
%\end{align}
%

%NUTRIENT
\subsection{Use Case 1}
The current model setup is based on the EMPOWER model \citep{Anderson2015c}, with modifications to:\\ 
- light harvesting\\ 
- light & nutrient $N_0$ forcing\\ 
- zooplankton mixing\\ 
- temperature dependence of phytoplankton growth?.\\ 

\\ 
The physical environment of the model uses slab physics as presented in \citet{Evans1985ACycles}. The model ocean is built up of a biologically inert deep ocean below a well mixed layer of seasonally variable depth that contains the ecosystem. The ecosystem component is a traditional NPZD model, containing four state variables describing nutrients $N$, phytoplankton $P$, zooplankton $Z$ and detritus $D$.
\subsubsection{Model Equations}
The rates of change of the state variables are described by the following set of equations:

\begin{equation}
    \frac{d N}{d t} = 
    K (N_0 - N) % Nutrient mixing
    + \beta(1 - k_{NZ})(G_P + G_D) % Unassimilated grazing by Z
    + m_D \cdot D % Remineralisation of D
    - \mu_{max}(T) \  \gamma^{I} \ \gamma^{N} \cdot P % Phytoplankton gains
\end{equation}

%PHYTOPLANKTON
\begin{equation}
    \frac{d P}{d t} =
    \mu_{max}(T) \  \gamma^{I} \ \gamma^{N} \cdot P  % Phytoplankton gains
    - m_P \cdot P % Linear mortality
    - m_{P2} \cdot P^2 % Quadratic mortality
    - G_P % Z grazing
    - K \cdot P % Phytoplankton mixing
\end{equation}

%ZOOPLANKTON
\begin{equation}
    \frac{d Z}{d t} =
    \beta k_{NZ}(G_P + G_D) % Assimilated grazing
    - m_Z \cdot Z % Linear mortality
    - m_{Z2} \cdot Z^2 % Quadratic mortality
    - K \cdot Z % Zooplankton mixing
\end{equation}

%DETRITUS
\begin{equation}
    \frac{d D}{d t} = 
    m_P \cdot P % Linear mortality
    + m_{P2} \cdot P^2 % Quadratic mortality
    + m_Z \cdot Z % Linear mortality
    + (1 - \beta)(G_P + G_D) % Unassimilated grazing by Z
    - G_D % Z grazing on D
    - m_D \cdot D % Remineralisation of D
    - K \cdot D % Mixing of D
    - \frac{v}{H(t)} \cdot D % Sinking of D
\end{equation}



\subsubsection{Mixing}

The mixing coefficient $K$ describes mixing across the bottom of the mixed layer.

\begin{equation}
    K = \frac{1}{H(t)} \cdot \left(h^{+}(t) + \kappa\right)
\end{equation}

Constant diffusive mixing is parameterized by $\kappa$, which is further modified by the changes in MLD. The MLD at a certain time point is $H(t)$ and the change in MLD is its derivative $\frac{d}{d t} H(t)$. The function $h^{+}(t)$ gives the effects of entrainment and detrainment due to the changes in MLD.

\begin{equation}
    h^{+}(t) = \max\left(0, \frac{d}{d t} H(t)\right)
\end{equation}

The derivative of MLD $\frac{d}{d t} H(t)$ is positive when the mixed layer deepens. Nutrients are entrained from below, while other components of the ecosystem (e.g. phytoplankton) are diluted. When the mixed layer shallows, $h^{+}(t)$ does not modify $K$ (i.e. returns 0 instead of a negative value). It is assumed that detrainment of mass and the increase in concentration due to the reduced volume of the mixed layer are balanced.
\\ !ToDO: mixing of Z is different \\
$K$ affects all state variables in the upper mixed layer. In addition to $K$, detritus experiences losses due to gravitational sinking at a rate of $v$. This term is added to describe the fast export of larger detritus particles below the mixed layer. 

\subsubsection{Phytoplankton}
Phytoplankton biomass $P$ increases through temperature-dependent, light- \& nutrient-limited growth. Temperature dependence of the maximum growth rate $\mu_{max}(T)$ is calculated from the Eppley curve \citep{Eppley1972TemperatureSea}.

\begin{equation}
    \mu_{max}(T) = V^{max}_P = V^{max}_P(0) \cdot 1.066^{T_{MLD}} \label{mumax}
\end{equation}

With the assumption of balanced growth the maximum growth rate $\mu_{max}$ is equal to the maximum photosynthetic rate $V^{max}_P$. The temperature forcing $T_{MLD}$ is the average temperature of the mixed layer.

The light limitation of phytoplankton growth is described by the Smith photosynthesis-irradiance (P-I) function:

\begin{equation}
    V_P = \frac{\alpha \cdot I \cdot V^{max}_P}{\sqrt{(V^{max}_P)^2 + \alpha^2 I^2}}
\end{equation}
where $V_P$ is the photosynthetic rate, $\alpha$ is the initial slope of the P-I curve, $I$ is irradiance and $V^{max}_P$ is the temperature-dependent maximum photosynthetic rate defined in equation \eqref{mumax}.


The irradiance forcing $I_0$ is a time averaged measurement of photosynthetically active radiation (PAR) at the surface. Attenuation of $I_0$ at depth $z$ in the mixed layer is calculated according to the Lambert-Beer equation:

\begin{equation}
    I(z) = I_0 \cdot e^{-k_{PAR} \cdot Z} \label{beer}
\end{equation}

The attenuation coefficient $k_{PAR}$ is the sum of the attenuation coefficient of seawater $k_w$ and that of chlorophyll $k_c$, which is multiplied by phytoplankton biomass $P$:

\begin{equation}
    k_{PAR} = k_w + k_c \cdot P
\end{equation}

Combining the equations and integrating across the mixed layer, the numerical solution for the integrated light-limiting term affecting phytoplankton growth is:

\begin{equation}
    \gamma^I = \frac{V^P_{max}}{k_{PAR} \cdot H(t)} \ln{ \left( \frac{ x_0+\sqrt{(V^{max}_P)^2+x_0^2} }{ x_z+\sqrt{(V^{max}_P)^2+x_z^2} } \right)}
\end{equation}
where $x_0$ = $\alpha \cdot I_0$ and $x_z$ = $\alpha \cdot I(H(t))$, with $I(z)$ calculated according to equation \eqref{beer}.

Nutrient limitation of phytoplankton growth $\gamma^N$ is described by the Michaelis-Menten (or Monod) equation.

\begin{equation}
    \gamma^N = \frac{N}{k_N + N}
\end{equation}

where $k_N$ is the half-saturation constant. $N$ is nutrient concentration, in this case dissolved inorganic nitrogen (DIN).

Non-grazing mortality of phytoplankton is described by both a linear $m_P$ and a quadratic factor $m_{P2}$. The former accounts for natural mortality and excretion. Quadratic mortality describes density-dependent loss processes, which can be caused by viral infection. All non-grazing mortality terms feed into the detritus pool

\subsubsection{Zooplankton}
Grazing by zooplankton occurs on both phytoplankton and detritus. The grazing function is a Holling Type 3 grazing response as presented in \citet{Anderson2015c}.

\begin{equation}
    G_P = I_{max} \left( \frac{ \hat{\phi}_P P}{k_Z^2 + \hat{\phi}_D D +\hat{\phi}_P P}  \right) Z
\end{equation}
where $\hat{\phi}_P$ = $\phi_P \cdot P$, $\hat{\phi}_D$ = $\phi_D \cdot D$.

This formulation describes the total biomass of phytoplankton that is grazed $G_P$. Parameter $I_{max}$ is the maximum specific ingestion rate for a food source. Here the value is the same for both phytoplankton and detritus. The density dependent feeding coefficients $\hat{\phi}$ are calculated from multiplying the feeding parameter $\phi$ by prey biomass. The actual relative feeding preference can be calculated from the ration between $\hat{\phi}_D$ and $\hat{\phi}_P$. The arbitrary parameter $k_Z$ determines the half-saturation constants $k$ for grazing on a specific prey based on the choice of $\phi$, with the relationship $k$ = $\sqrt{\frac{k^2_Z }{ \phi}}$

For detritus the formula is the following:

\begin{equation}
    G_P = I_{max} \left( \frac{ \hat{\phi}_D D}{k_Z^2 + \hat{\phi}_D D +\hat{\phi}_P P}  \right) Z
\end{equation}

The sigmoidal response includes passive prey switching via an interference effect, where the increase in biomass of one prey slightly reduces the intake of other prey. The effect does not create sub-optimal feeding (i.e. an increase in biomass of a less common prey decreases total grazing), which can be observed in active switching grazing formulations as that used by \citet{Fasham1990a}.

Zooplankton ingestion of prey does not directly convert to biomass gained however. The total biomass grazed $G$ is split three ways between growth, excretion of dissolved nutrients and egestion of faecal matter \& particles. The parameters describing the contribution of each process are the absorption efficiency $\beta$ and net production efficiency $k_{NZ}$. Gross growth efficiency is the product of these two factors.  

Similar to phytoplankton mortality, the linear mortality factor $m_Z$ parameterizes natural mortality and excretion and feeds into the pool of detritus. The quadratic factor $m_{Z2}$ describes higher order predation on zooplankton and is removed from the system.



\subsubsection{Model Parameters}
The model parameters in Table \ref{appendix:table:usecase1parameters} are adapted from \citet{Anderson2015c}.


\subsubsection{NOTES:}
Yes, the $V^{max}_P$ \& $\alpha$  units are pretty weird, and do require a unit conversion to $d^{-1}$ and the result of the light harvesting function needs to be divided by $\theta_{chla}$ to convert to the actual non-dimensional light limiting term. To me it seems like this is not well explained in the EMPOWER publication (still doesn't REALLY make sense to me), but I actually saw how they calculated it in the model code. I still need to work on how to present that understandably in the text before. 
\\ !ToDO: 
- make sure that all parameters have a uniform description, and then update accordingly in table, then update accordingly in text! \\



\clearpage
%TABLES & FIGURES



\begin{table*}[t]
\caption{Definition of symbols employed in use case 1 appendix, with the corresponding units. \unit{µM \ N} = \unit{mmol \ Nitrogen \ m^{-3}}}
\begin{tabular}{l l l}
Symbol & Meaning & Unit\\
\tophline
$N$ & concentration of nutrient in the upper mixed layer & \unit{µM \ N} \\
$P$ & concentration of phytoplankton biomass in the upper mixed layer & \unit{µM \ N} \\
$Z$ & concentration of zooplankton biomass in the upper mixed layer & \unit{µM \ N} \\
$D$ & concentration of detritus in the upper mixed layer & \unit{µM \ N} \\
$K$ & material exchange between mixed and bottom layer & \unit{d^{-1}} \\
$N_0$ & nutrient concentration right below mixed layer & \unit{µM \ N} \\
% TODO: change all square brackets to \unit{  }
$\beta$ & absorption efficiency of zooplankton grazing &  dimensionless \\
$k_{NZ}$ & net production efficiency of zooplankton grazing & dimensionless \\
$G_P$ & total biomass of phytoplankton grazed by zooplankton & \unit{µM \ N} \\
$G_D$ & total biomass of detritus grazed by zooplankton & \unit{µM \ N} \\
$m_D$ & remineralisation rate of detritus & \unit{d^{-1}} \\
$\mu_{max}(T)$ & temperature-dependent max. growth rate of phytoplankton & \unit{d^{-1}} \\
$\gamma_I$ & light limitation of phytoplankton growth &  dimensionless\\
$\gamma_N$ & nutrient limitation of phytoplankton growth & dimensionless \\
$m_P$ & linear mortality of phytoplankton & \unit{d^{-1}} \\
$m_{P2}$ & quadratic mortality of phytoplankton & \unit{(\mu M \ N)^{-1} \ d^{-1}} \\
$m_Z$ & linear mortality of zooplankton & \unit{d^{-1}} \\
$m_{Z2}$ & quadratic mortality of zooplankton & \unit{(\mu M \ N)^{-1} \ d^{-1}} \\
$v$ & sinking rate of detritus & \unit{m \ d^{-1}}\\
%\middlehline
%\bottomhline
\end{tabular}
%\belowtable{This is a test} % Table Footnotes
\end{table*}



\begin{table*}[t]
\caption{Model Parameters}
\begin{tabular}{l l l l l}
Parameter & Meaning & Unit & Temperate & Tropical \\
\tophline
$V^{max}_P$ & max photosynthesis rate & \unit{g \ C (g \ chl)^{−1} h^{−1}} & 2.5  &  \\
$\alpha$ & initial slope of P-I curve & \unit{g \ C (g \ chl)^{−1} h^{−1} (µE \ m^{-2} \ s^{-1})^{-1}} & 0.034 & \\
$k_N$ & half-sat. const: N uptake & \unit{µM \ N} & 0.85 & \\
$m_P$ & linear P mortality & \unit{d^{−1}} & 0.015 & \\
$m_{P2} & quadratic P mortality & \unit{(µM \ N)^{-1} d^{−1}} & 0.025 & \\
$I_{max} & Z max. ingestion rate & \unit{d^{−1}} & 1.0 & \\
$k_Z$ & Z half-saturation for intake & \unit{µM \ N} & 0.6 & \\
$\phi_P$ & grazing preference: P & 0.67 & & \\
$\phi_D$ & grazing preference: D & & 0.33 & \\
$\beta_Z$ & Z absorption efficiency & 0.69 & & \\
$k_{NZ} & Z net production efficiency & 0.75 & & \\
$m_Z$ & linear Z mortality  & \unit{d^{−1}} & 0.02 & \\
$m_{Z2} & quadratic Z mortality & \unit{(µM \ N)^{-1} d^{−1}} & 0.34 & \\
$v_D$ & D sinking rate & \unit{m \ d^{−1}} & 6.43 & \\
$m_D$ & D remineralisation rate & \unit{d^{−1}} & 0.06 & \\
$\kappa$ & constant mixing parameter & \unit{m \ d^{−1}} & 0.13 & \\
$\theta_{chla}$ & C to chlorophyll ratio & \unit{g \ g^{-1}} & 75 & \\
\middlehline

\bottomhline
\end{tabular}
\belowtable{The units for light harvesting are not final! I will most likely use mol photons, or whatever unit is SI and you all like.} % Table Footnotes
\label{appendix:table:usecase1parameters}
\end{table*}
%


\clearpage

% this is a custom function to be able to see references when rendering subfiles:
\biblio

\end{document}
