%%% TABLES %%%%

\begin{table*}[t]
\caption{Allometric parameterisations and empirical parameter values employed in use case 2, adapted from \citet{Banas2011b}}
\begin{tabular}{l l l l l}
Empirical fit & Applicability & Source \\
\tophline
$\mu^i_{0} = (2.6 \ d^{-1}) \left( \frac{size^i_{P}}{1\mu m} \right)^{-0.45}$ & Phytoplankton 1-100 ESD \unit{\mu m} & Tang(1995) \\
$k^i_N = (0.1 \ \unit{\mu M \ N})\left( \frac{size^i_{P}}{1\mu m} \right)$ & Phytoplankton 1-100 ESD \unit{\mu m} & Eppley et al. (1969) \\

$I^j_0 = (26 \ d^{-1})\left( \frac{size^i_{P}}{1\mu m} \right)^{-0.4}$ & Flagellates, dinoflagellates, ciliates, copepods & Hansen et al. (1997) \\

$k^j_Z = 3 \ \unit{\mu M \ N} $ & Flagellates, dinoflagellates, ciliates, copepods & Hansen et al. (1997) \\

$size^j_{opt} = (0.65 \ \unit{\mu m})\left( \frac{size^i_{P}}{1\mu m} \right)^{0.56}$ & Flagellates, dinoflagellates, ciliates, copepods & Hansen et al. (1994) \\
$\Delta size_{P} = 0.25 $ & Ciliates, nauplii, copepodites & Hansen et al. (1994)  \\
\middlehline

\bottomhline
\end{tabular}
\belowtable{TODO: the sources need to be added properly! for now just text..} % Table Footnotes
\label{appendix:table:usecase2parameters}
\end{table*}