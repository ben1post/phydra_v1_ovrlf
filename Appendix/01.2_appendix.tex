\documentclass[template.tex]{subfiles}


\begin{document}


\subsection{Use Case 2}
The model setup is based on the ASTroCAT model \citep{Banas2011b}, with minor modifications to the physical environment. 
Our model setup describes a size-structured community of phytoplankton and zooplankton, with a single nutrient resolved, embedded in a physical setting akin to a laboratory chemostat.
A chemostat is a flow-through system, where organisms are grown in a bottle with a continuous influx of a nutrient solution. At the same rate all components of the bottle flow out of the bottle.

The ecosystem component is a NPZ model, containing three components describing nutrients $N$, phytoplankton $P$, zooplankton $Z$. In addition this model describes multiple state variables of phytoplankton and zooplankton, with each state variable defined by their equivalent spherical diameter.
In model runs presented in this paper, we used follow \citet{Banas2011b} in running simulations with 40 size classes of each planktonic component. 
The technical implementation allows running the model with any number of size classes, therefore the mathematical description will be kept general in describing the number of size classes by the subscript $i$ for phytoplankton and $j$ for zooplankton. Note that the implementation of a size-spectral model by \citeauthor{Banas2011b} implicitly has the same number of phytoplankton and zooplankton state variables (i.e. $i = j$).



\subsubsection{Model Equations}
The rates of change of the state variables are described by the following set of equations:

\begin{equation}
    \frac{d N}{d t} = 
    f \cdot N^0 % Nutrient mixing
    + (1 - \epsilon - \beta)\sum_{j} \sum_{i} G_{ij}^P % Unassimilated grazing by Z
    - \sum_{i} ( \mu_i^{\emptyset} \cdot \gamma_i^N \cdot P_i) % Phytoplankton gains
    - f \cdot N
\end{equation}

%PHYTOPLANKTON
\begin{equation}
    \frac{d P_i}{d t} =
    \mu_i^{\emptyset} \cdot  \gamma_i^N \cdot   P_i  % Phytoplankton gains
    - m^P  \cdot \mu_i^{0} \cdot P_i % Linear mortality
    - \sum_{j} G_{ij}^P % Z grazing
    - f \cdot P_i
\end{equation}

%ZOOPLANKTON
\begin{equation}
    \frac{d Z_j}{d t} =
    \epsilon \cdot \sum_{i} G_{ij}^P % Assimilated grazing
    - m^{Z2} \cdot Z_j \cdot \sum_{j} Z_j  % Quadratic mortality
    - f \cdot Z_j
\end{equation}



\subsubsection{Environment}

Nutrient supply to the system is defined by a linear flow rate $f$ and the nutrient concentration in the supplied solution $N^0$. All components, including phytoplankton and zooplankton are lost from the system at the same rate $f$.

\subsubsection{Phytoplankton}
Phytoplankton biomass $P$ increases through  nutrient-limited growth. 

Nutrient limitation of phytoplankton growth $\gamma_i^N$ is described by the Michaelis-Menten (or Monod) equation.

\begin{equation}
    \gamma^i_N =  \frac{N}{k^i_N + N} 
\end{equation}

where $k_i^N$ is the size-dependent half-saturation constant. $N$ is nutrient concentration, in this case dissolved inorganic nitrogen (DIN).


Non-grazing mortality of phytoplankton is described the factor $m^P$ that is scaled by the maximum intrinsic growth rate $\mu_i^{0}$. This accounts for natural mortality and excretion.

\subsubsection{Zooplankton}
Zooplankton size class $j$ grazing on phytoplankton size class $i$ is calculated by
\begin{equation}
    G_{ij}^P = I_j^0 \ \frac{ \phi_{ij} \cdot P_i }{ k_j^Z + \sum_{i}(\phi_{ij} \cdot P_i) } \ Z_j
\end{equation}
where $I_j^0$ is the size-dependent maximum ingestion rate, $k_j^Z$ is the prey half-saturation level and $\phi_{ij}$ is the relative preference of $Z_j$ for prey type $P_i$.\\

Prey preference is assumed to vary with phytoplankton size $size_i^{P}$ in a log-Gaussian distribution around an optimal prey size for each grazer $size_j^{opt}$.

\begin{equation}
    \phi_{ij} = exp \left[ -\left( \ \frac{ log_{10}(size_i^{P}) - log_{10}(size_j^{opt}) }{ \Delta size^{P} } \right) \right]
\end{equation}
Where $\Delta size^{P}$ is the prey size tolerance parameter, with units of \unit{log_{10}(\mu m)}, that controls the width of the Gaussian distribution.\\

\subsubsection{Model Parameters}
The model parameters in table \ref{appendix:table:usecase2parameters}.


\subsubsection{NOTES:}
...

\clearpage
%TABLES & FIGURES



\begin{table*}[t]
\caption{Definition of symbols employed in use case 2 appendix, with the corresponding units. \unit{µM \ N} = \unit{mmol \ Nitrogen \ m^{-3}}}
\begin{tabular}{l l l}
Symbol & Meaning & Unit\\
\tophline
$N$ & concentration of nutrient in the upper mixed layer & \unit{µM \ N} \\
$P_i$ & concentration of phytoplankton size class $i$ biomass in the upper mixed layer & \unit{µM \ N} \\
$Z_j$ & concentration of zooplankton size class $i$ biomass in the upper mixed layer & \unit{µM \ N} \\
$D$ & concentration of detritus in the upper mixed layer & \unit{µM \ N} \\
$f$ & flow rate of chemostat flow-through & \unit{d^{-1}} \\
$N_0$ & nutrient concentration of source medium for chemostat & \unit{µM \ N} \\
% TODO: change all square brackets to \unit{  }
$\beta$ & absorption efficiency of zooplankton grazing &  dimensionless \\
$\epsilon$ & net production efficiency of zooplankton grazing & dimensionless \\
$G_{ij}^P$ & total biomass of phytoplankton size class $j$ grazed by zooplankton size class $i$ & \unit{µM \ N} \\
$\mu_i^{0}$ & size-dependent max. growth rate of phytoplankton & \unit{d^{-1}} \\
$\gamma_i^N$ & nutrient-dependence of the growth of phytoplankton size class $i$ & dimensionless\\
$m^P$ & fraction of $\mu_i^{0}$ that is phytoplankton mortality & dimensionless \\
$m^{Z2}$ & quadratic mortality of zooplankton, dependent on total $Z$ biomass & \unit{(\mu M \ N)^{-1} \ d^{-1}} \\
$k_i^N$ & Phytoplankton size class $i$ half-saturation constant for nutrient uptake & \unit{µM \ N} \\
$I_j^0$ & maximum ingestion rate for zooplankton size class $j$ &  \unit{d^{-1}} \\
$\phi_{ij}$ & Preference of grazer $Z_j$ for prey $P_i$ & dimensionless \\
$k_j^Z$ & Prey half-saturation level for zooplankton class $j$ & \unit{µM \ N} \\
$size_i^{P}$ & Individual size in phytoplankton (prey) class $i$ & \unit{\mu m} \\
$size_j^{opt}$ & Optimal prey size for zooplankton size class $j$ & \unit{\mu m} \\
$\Delta size^{P}$ & Prey size tolerance for grazers & \unit{log_{10}(\mu m)} \\
%\middlehline
%\bottomhline
\end{tabular}
%\belowtable{This is a test} % Table Footnotes
\label{appendix:table:usecase2symbols}
\end{table*}



\begin{table*}[t]
\caption{Allometric parameterisations and empirical parameter values employed in use case 2, adapted from \citet{Banas2011b}.}
\begin{tabular}{l l l l l}
Empirical fit & Applicability & Source \\
\tophline
$\mu_i^{0} = (2.6 \ d^{-1}) \left( \frac{size^i_{P}}{1\mu m} \right)^{-0.45}$ & Phytoplankton 1-100 ESD \unit{\mu m} & Tang(1995) \\
$k_i^N = (0.1 \ \unit{µM \ N})\left( \frac{size^i_{P}}{1\mu m} \right)$ & Phytoplankton 1-100 ESD \unit{\mu m} & Eppley et al. (1969) \\

$I_j^0 = (26 \ d^{-1})\left( \frac{size^i_{P}}{1\mu m} \right)^{-0.4}$ & Flagellates, dinoflagellates, ciliates, copepods & Hansen et al. (1997) \\

$k_j^Z = 3 \ \unit{µM \ N} $ & Flagellates, dinoflagellates, ciliates, copepods & Hansen et al. (1997) \\

$size_j^{opt} = (0.65 \ \unit{\mu m})\left( \frac{size^i_{P}}{1\mu m} \right)^{0.56}$ & Flagellates, dinoflagellates, ciliates, copepods & Hansen et al. (1994) \\
$\Delta size^{P} = 0.25 $ & Ciliates, nauplii, copepodites & Hansen et al. (1994)  \\
\middlehline

\bottomhline
\end{tabular}
\belowtable{TODO: the sources need to be added properly! for now just text..} % Table Footnotes
\label{appendix:table:usecase2parameters}
\end{table*}
%


\clearpage

% this is a custom function to be able to see references when rendering subfiles:
\biblio

\end{document}
