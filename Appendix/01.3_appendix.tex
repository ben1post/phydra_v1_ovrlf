\documentclass[template.tex]{subfiles}


\begin{document}


\subsection{Use Case 3}
The model setup is a combination of the previous two use cases. The size-structured phytoplankton and zooplankton components of use case 2 are placed within the physical slab setting of use case 1. Components and processes are adapted from both \citet{Banas2011b} and \citet{Anderson2015c}.

Model setup describes a size-structured community of phytoplankton and zooplankton, with a single nutrient resolved, embedded in a well-mixed upper layer.

The ecosystem component is a NPZD model, containing three components describing nutrients $N$, phytoplankton $P$, zooplankton $Z$ and detritus $D$. In addition this model describes multiple state variables of phytoplankton and zooplankton, with each state variable defined by their equivalent spherical diameter (ESD).

The technical implementation allows running the model with any number of size classes, therefore the mathematical description will be kept general in describing the number of size classes by the subscript $i$ for phytoplankton and $j$ for zooplankton. Note that the implementation of a size-spectral model by \citeauthor{Banas2011b} implicitly has the same number of phytoplankton and zooplankton state variables (i.e. $i = j$).

"The model implementation discussed here includes 40 equally log-spaced classes of P from 1 to 20 \unit{\mu m}, and 40 matching classes of Z from 2.1 to 460 \unit{\mu m}. The $size^{opt}_j$ power law is used to choose the zooplankton size classes, so that every Z class has its optimal prey size available. It is also important to insure that overall range of P and Z sizes match in this sense, to avoid creating classes of organisms at the small or large ends of the spectrum that are artificially released from grazing pressure or artificially suppressed."


\subsubsection{Model Equations}
The rates of change of the state variables are described by the following set of equations:

%%%%%%%%%%%%%%%%%%%%%%%%%%%%%%%%%%%%%%%%%%%%%%%%%%%%%%%%%%%%%%%%%

\begin{equation}
    \frac{d N}{d t} = 
    K (N^0 - N) % Nutrient mixing
    +  \beta(1 - \epsilon)\sum_{j} \sum_{i} G_{ij}^P % Unassimilated grazing by Z
    +  m^D \cdot D % Remineralisation of D
    - \sum_{i} ( \mu_i^{0} \  \gamma^{T} \ \gamma^{I} \ \gamma_i^N \cdot P_i) % Phytoplankton gains
\end{equation}

%PHYTOPLANKTON
\begin{equation}
    \frac{d P_i}{d t} =
    \mu_i^{0} \ \gamma^{T} \ \gamma^{I} \ \gamma_i^N \cdot   P_i  % Phytoplankton gains
    - m^P  \ \mu_i^{0} \cdot P_i % Linear mortality
    - \sum_{j} G_{ij}^P % Z grazing
    - K \cdot P_i% Phytoplankton mixing
\end{equation}

%ZOOPLANKTON
\begin{equation}
    \frac{d Z_j}{d t} =
    \beta \ \epsilon \ \sum_{i} G_{ij}^P % Assimilated grazing
    - m^{Z2} \cdot Z_j \cdot \sum_{j} Z_j  % Quadratic mortality
    - K \cdot Z_j % Phytoplankton mixing
\end{equation}

%DETRITUS
\begin{equation}
    \frac{d D}{d t} = 
    \sum_{i}( m^P  \ \mu_i^{0} \cdot P_i) % Linear mortality
    + (1 - \beta) \sum_{j} \sum_{i} G_{ij}^P % Unassimilated grazing by Z
    - m^D \cdot D % Remineralisation of D
    - K \cdot D % Mixing of D
    - \frac{v}{H(t)} \cdot D % Sinking of D
\end{equation}


\subsubsection{Environment}



\subsubsection{Phytoplankton}
Phytoplankton biomass $P$ increases through  nutrient-limited growth. 

Nutrient limitation of phytoplankton growth $\gamma^i_N$ is described by the Michaelis-Menten (or Monod) equation.

\begin{equation}
    \gamma_i^N =  \frac{N}{k_i^N + N} 
\end{equation}

where $k^i_N$ is the size-dependent half-saturation constant. $N$ is nutrient concentration, in this case dissolved inorganic nitrogen (DIN).


Non-grazing mortality of phytoplankton is described the factor $m_P$ that is scaled by the maximum intrinsic growth rate $\mu^i_{0}$. This accounts for natural mortality and excretion.

\subsubsection{Zooplankton}
Zooplankton size class $j$ grazing on phytoplankton size class $i$ is calculated by
\begin{equation}
    G_{ij}^P = I_j^0 \ \frac{ \phi_{ij} \cdot P_i }{ k_j^Z + \sum_{i}(\phi_{ij} \cdot P_i) } \ Z_j
\end{equation}
where $I_j^0$ is the size-dependent maximum ingestion rate, $k_j^Z$ is the prey half-saturation level and $\phi_{ij}$ is the relative preference of $Z_j$ for prey type $P_i$.\\

Prey preference is assumed to vary with phytoplankton size $size_i^{P}$ in a log-Gaussian distribution around an optimal prey size for each grazer $size_j^{opt}$.

\begin{equation}
    \phi_{ij} = exp \left[ -\left( \ \frac{ log_{10}(size_i^{P}) - log_{10}(size_i^{opt}) }{ \Delta size^{P} } \right) \right]
\end{equation}
Where $\Delta size^{P}$ is the prey size tolerance parameter, with units of \unit{log_{10}(\mu m)}, that controls the width of the Gaussian distribution.\\

Zooplankton ingestion of prey does not directly convert to biomass gained however. The total biomass grazed $G$ is split three ways between growth, excretion of dissolved nutrients and egestion of faecal matter \& particles. The parameters describing the contribution of each process are the absorption efficiency $\beta$ and net production efficiency $\epsilon$. Gross growth efficiency is the product of these two factors.  \\

\subsubsection{Model Parameters}
The model parameters in table \ref{appendix:table:usecase2parameters}.


\subsubsection{NOTES:}
...

\clearpage
%TABLES & FIGURES



\begin{table*}[t]
\caption{Definition of symbols employed in use case 3 appendix, with the corresponding units. \unit{µM \ N} = \unit{mmol \ Nitrogen \ m^{-3}}}
\begin{tabular}{l l l}
Symbol & Meaning & Unit\\
\tophline
$N$ & concentration of nutrient in the upper mixed layer & \unit{µM \ N} \\
$P_i$ & concentration of phytoplankton size class $i$ biomass in the upper mixed layer & \unit{µM \ N} \\
$Z_j$ & concentration of zooplankton size class $i$ biomass in the upper mixed layer & \unit{µM \ N} \\
$D$ & concentration of detritus in the upper mixed layer & \unit{µM \ N} \\
$f$ & flow rate of chemostat flow-through & \unit{d^{-1}} \\
$N_0$ & nutrient concentration of source medium for chemostat & \unit{µM \ N} \\
% TODO: change all square brackets to \unit{  }
$\beta$ & absorption efficiency of zooplankton grazing &  dimensionless \\
$\epsilon$ & net production efficiency of zooplankton grazing & dimensionless \\
$G_{ij}^P$ & total biomass of phytoplankton size class $j$ grazed by zooplankton size class $i$ & \unit{µM \ N} \\
$\mu_i^{\emptyset}$ & size-dependent max. growth rate of phytoplankton & \unit{d^{-1}} \\
$\gamma^i_N$ & nutrient-dependence of the growth of phytoplankton size class $i$ & dimensionless\\
$m_P$ & fraction of $\mu^i_{0}$ that is phytoplankton mortality & dimensionless \\
$m_{Z2}$ & quadratic mortality of zooplankton, dependent on total $Z$ biomass & \unit{(\mu M \ N)^{-1} \ d^{-1}} \\
$k^i_N$ & Phytoplankton size class $i$ half-saturation constant for nutrient uptake & \unit{µM \ N} \\
$I^j_0$ & maximum ingestion rate for zooplankton size class $j$ &  \unit{d^{-1}} \\
$\phi_{ij}$ & Preference of grazer $Z_j$ for prey $P_i$ & dimensionless \\
$k^j_Z$ & Prey half-saturation level for zooplankton class $j$ & \unit{µM \ N} \\
$size^i_{P}$ & Individual size in phytoplankton (prey) class $i$ & \unit{\mu m} \\
$size^j_{opt}$ & Optimal prey size for zooplankton size class $j$ & \unit{\mu m} \\
$\Delta size_{P}$ & Prey size tolerance for grazers & \unit{log_{10}(\mu m)} \\
%\middlehline
%\bottomhline
\end{tabular}
%\belowtable{This is a test} % Table Footnotes
\label{appendix:table:usecase2symbols}
\end{table*}



\begin{table*}[t]
\caption{Allometric parameterisations and empirical parameter values employed in use case 2, adapted from \citet{Banas2011b}}
\begin{tabular}{l l l l l}
Empirical fit & Applicability & Source \\
\tophline
$\mu^i_{0} = (2.6 \ d^{-1}) \left( \frac{size^i_{P}}{1\mu m} \right)^{-0.45}$ & Phytoplankton 1-100 ESD \unit{\mu m} & Tang(1995) \\
$k^i_N = (0.1 \ \unit{µM \ N})\left( \frac{size^i_{P}}{1\mu m} \right)$ & Phytoplankton 1-100 ESD \unit{\mu m} & Eppley et al. (1969) \\

$I^j_0 = (26 \ d^{-1})\left( \frac{size^i_{P}}{1\mu m} \right)^{-0.4}$ & Flagellates, dinoflagellates, ciliates, copepods & Hansen et al. (1997) \\

$k^j_Z = 3 \ \unit{µM \ N} $ & Flagellates, dinoflagellates, ciliates, copepods & Hansen et al. (1997) \\

$size^j_{opt} = (0.65 \ \unit{\mu m})\left( \frac{size^i_{P}}{1\mu m} \right)^{0.56}$ & Flagellates, dinoflagellates, ciliates, copepods & Hansen et al. (1994) \\
$\Delta size_{P} = 0.25 $ & Ciliates, nauplii, copepodites & Hansen et al. (1994)  \\
\middlehline

\bottomhline
\end{tabular}
\belowtable{TODO: the sources need to be added properly! for now just text..} % Table Footnotes
\label{appendix:table:usecase2parameters}
\end{table*}
%


\clearpage

% this is a custom function to be able to see references when rendering subfiles:
\biblio

\end{document}
